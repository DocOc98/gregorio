% !TEX program = LuaLaTeX+se

% This is a simple template for a LuaLaTeX document using gregorio scores.

\documentclass[
  fontsize=11pt,
  paper=a4
]{scrartcl} % document class: manual at https://ctan.org/pkg/koma-script

% load packages:
\usepackage[osf,p]{libertinus} % Decent (free) font, but should be changed if you have high standards
\usepackage{gregoriotex} % for gregorio score inclusion
\usepackage[latin]{babel} % set language

\setkomafont{section}{\normalfont\centering\huge\scshape} % section heading style
\setcounter{secnumdepth}{-\maxdimen} % remove section numbering

\begin{document}

% The title:
\section{Populus Sion}

% Here we set the space around the initial.
% Please report to http://gregorio-project.github.io/gregoriotex/details.html for more details and options
\grechangedim{beforeinitialshift}{2.2mm}{scalable}
\grechangedim{afterinitialshift}{2.2mm}{scalable}

% Here we set the initial font. Change 43 if you want a bigger initial.
\grechangestyle{initial}{\fontsize{43}{43}\selectfont}%

% We set red lines here, comment it if you want black ones.
\gresetlinecolor{gregoriocolor}

% We set VII above the initial manually
\grechangestyle{annotation}{\small\bfseries}
\greannotation{Intr.}
\greannotation{\textsc{vii}}

% We use the "commentary" field of the score in the top right corner:
\gresetheadercapture{commentary}{grecommentary}{string}

% and finally we include the scores. The file must be in the same directory as this one.
\gregorioscore[a]{PopulusSion}

\section{Factus est}

\gregorioscore[a]{FactusEst}

\end{document}
