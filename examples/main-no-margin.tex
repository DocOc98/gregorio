% !TEX program = LuaLaTeX+se

% This document produces a pdf containing the score with no margin.
% It is based on this solution:
% http://tex.stackexchange.com/questions/291630/best-way-to-know-bottom-line-height-of-a-pdf/291664#291664
%
% It contains advanced commands quite difficult to understand, but should be straightforward
% to adapt to your needs.
%
\documentclass[11pt]{article}
\usepackage{fontspec}
\usepackage{graphicx}
\usepackage{gregoriotex}
\usepackage{libertine}
% Do not load geometry, fullpage, or any package changing width
% or margins.

\hoffset-1in
\voffset-1in
\newbox\scorebox

% using the "commentary" header of the gabc file
\gresetheadercapture{commentary}{grecommentary}{string}

\begin{document}

% here the final "9" of the commentary would be cut too sharp. The reason is that
% the italic "9" glyph in the libertine font is larger than its nominal width
% (which is normal for italic), so we have to push the commentary left a little.
% We use an admittedly strange trick to do so, redefining grecommentary.
% You have to adapt the value 0.3mm to your configuration.
\let\grevanillacommentary\grecommentary
\def\grecommentary#1{\grevanillacommentary{#1\kern 0.3mm}}

\setbox\scorebox=\vbox{\hsize=10cm\relax % change the width of the score here
  \gregorioscore[a]{PopulusSion}
}
\pdfpagewidth\wd\scorebox
\pdfpageheight\dimexpr\ht\scorebox+\dp\scorebox\relax
\shipout\box\scorebox
\end{document}
