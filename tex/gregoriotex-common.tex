%GregorioTeX common file (stuff needed by both gregoriotex and gregoriosyms)
%
% Copyright (C) 2007-2019 The Gregorio Project (see CONTRIBUTORS.md)
%
% This file is part of Gregorio.
%
% Gregorio is free software: you can redistribute it and/or modify
% it under the terms of the GNU General Public License as published by
% the Free Software Foundation, either version 3 of the License, or
% (at your option) any later version.
%
% Gregorio is distributed in the hope that it will be useful,
% but WITHOUT ANY WARRANTY; without even the implied warranty of
% MERCHANTABILITY or FITNESS FOR A PARTICULAR PURPOSE.  See the
% GNU General Public License for more details.
%
% You should have received a copy of the GNU General Public License
% along with Gregorio.  If not, see <http://www.gnu.org/licenses/>.

\gre@declarefileversion{gregoriotex-common.tex}{5.2.1}% GREGORIO_VERSION

\ifnum\luatexversion<76%
  \gre@error{Error: this document must be compiled with LuaTeX (lualatex) 0.76 or later}%
\fi%

%%%%%%%%%
%% Debugging
%%%%%%%%%

\def\gre@debugmsg#1#2{%
  \IfStrEq{\gre@debug}{}{}%
    %false
    {\IfStrEq{#1}{all}%
      %true
      {\gre@error{Debug error: ‘all’ is not a permitted keyword}}%
      %false
      {\IfStrEq{\gre@debug}{,all,}%
        %true
        {\gre@typeout{GregorioTeX debug: (#1) #2}}%
        %false
        {\IfSubStr{\gre@debug}{,#1,}%
          %true
          {\gre@typeout{GregorioTeX debug: (#1) #2}}%
          %false
          {\relax}%
        }%
      }%
    }%
}%

\def\gre@fixdebug{%
  {% we're starting a group in order to localize \exploregroups
    \exploregroups%
    \StrRemoveBraces{\gre@debug}[\gre@debug]%
    \xdef\gre@debug{,\gre@debug,}%
  }%
}%

\IfStrEq{\gre@debug}{}{% then
}{% else
  \typeout{GregorioTeX is in debug mode}%
  \typeout{\gre@debug\space messages will be printed to the log.}%
  \gre@fixdebug\relax %
}%

%%%%%%%%%%%%%
% Macros to handle deprecation path.
%%%%%%%%%%%%%

% #1 - the deprecated macro
% #2 - the correct macro to use
\def\gre@deprecated#1#2{%
  \ifgre@allowdeprecated%
    \gre@warning{#1\space is deprecated.\MessageBreak Use #2\space instead}%
  \else%
    \gre@error{#1\space is deprecated.\MessageBreak Use #2\space instead}%
  \fi%
  \relax%
}%

\def\gre@obsolete#1#2{%
  \gre@error{#1\space is obsolete.\MessageBreak Use #2\space instead}%
  \relax%
}%

%%%%%%%%%%%%%
% Internal function tracing
%%%%%%%%%%%%%

% Each function in the internal name space (\gre@...) should have an invocation of
% \gre@trace as its first line with the argument being the name of the function (without
% the backslash) and the list of arguments.
% The last line of these functions should be \gre@trace@end.
% Macros which merely store a value are not considered to be functions and thus not
% included in the trace stack.
% We also exempt the internal aliases for functions defined elsewhere.
% The tracing messages are turned on via the debug interface, keyword: trace.

\def\gre@trace@prefix{GreTrace: }
\def\gre@trace#1{%
  \IfSubStr{\gre@debug}{,trace,}%
    {%print tracing messages and increase indent
      \gre@typeout{\gre@trace@prefix \unexpanded{#1}}%
      \edef\gre@trace@prefix{\gre@trace@prefix| }%
    }{}%do nothing if not tracing
}%
\def\gre@trace@end{%
  \IfSubStr{\gre@debug}{,trace,}%
    {%dedent trace messages and mark end
      \StrGobbleRight{\gre@trace@prefix}{2}[\gre@trace@prefix]%
      \gre@typeout{\gre@trace@prefix -----}%
    }{}%do nothing if not tracing
}

