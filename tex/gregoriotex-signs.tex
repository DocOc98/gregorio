%GregorioTeX file.
%
% Copyright (C) 2007-2017 The Gregorio Project (see CONTRIBUTORS.md)
%
% This file is part of Gregorio.
%
% Gregorio is free software: you can redistribute it and/or modify
% it under the terms of the GNU General Public License as published by
% the Free Software Foundation, either version 3 of the License, or
% (at your option) any later version.
%
% Gregorio is distributed in the hope that it will be useful,
% but WITHOUT ANY WARRANTY; without even the implied warranty of
% MERCHANTABILITY or FITNESS FOR A PARTICULAR PURPOSE.  See the
% GNU General Public License for more details.
%
% You should have received a copy of the GNU General Public License
% along with Gregorio.  If not, see <http://www.gnu.org/licenses/>.


% this file contains definitions of signs (bar, episema, punctum, alterations)

\def\grebarbracewidth{.58879}%

\gre@debugmsg{general}{Loading gregoriotex-signs.tex}%
\gre@declarefileversion{gregoriotex-signs.tex}{5.0.0-beta2}% GREGORIO_VERSION

\def\gre@usestylecommon{%
  \ifgre@usestylefont\else %
    \gre@usestylefonttrue%
    \gre@setstylefont %
  \fi %
  \relax %
}%

\gresetglyphstyle{default}%

% Possibility to disable some signs:
\newif\ifgre@disablevepisema
\gre@disablevepisemafalse
\newif\ifgre@disablehepisema
\gre@disablehepisemafalse
\newif\ifgre@disablemora
\gre@disablemorafalse

\def\greprintsigns#1#2{%
  \IfStrEqCase{#2}{%
    {enable}%
      {%
        \IfStrEqCase{#1}{%
          {vepisema}%
            {\global\gre@disablevepisemafalse}%
          {hepisema}%
            {\global\gre@disablehepisemafalse}%
          {mora}%
            {\global\gre@disablemorafalse}%
          {all}%
            {\global\gre@disablemorafalse\global\gre@disablevepisemafalse\global\gre@disablehepisemafalse}%
          }[% all other cases
            \gre@error{Unrecognized first argument for \protect\greprintsigns}%
          ]%
      }%
    {disable}%
      {%
        \IfStrEqCase{#1}{%
          {vepisema}%
            {\global\gre@disablevepisematrue}%
          {hepisema}%
            {\global\gre@disablehepisematrue}%
          {mora}%
            {\global\gre@disablemoratrue}%
          {all}%
            {\global\gre@disablemoratrue\global\gre@disablevepisematrue\global\gre@disablehepisematrue}%
          }[% all other cases
            \gre@error{Unrecognized first argument for \protect\greprintsigns}%
          ]%
      }%
    }[% all other cases
      \gre@error{Unrecognized second argument for \protect\greprintsigns}%
    ]%
}%

%%%%%%%%%%%%%%%%%%%%%%%%%%%%%
%% macros for discretionaries
%%%%%%%%%%%%%%%%%%%%%%%%%%%%%
%
% In order to avoid clef change at beginning or end of line, we use discretionaries
% for clef change, or even with more complex data (z0::c3 for instance). The problem
% with discretionaries is that:
% - you cannot use \hskip (but you can use kern)
% - you cannot use \penalty (which is useless indeed)
%
% To remedy that, we define \gre@hskip to be \hskip outside a discretionary, and
% \kern inside a discretionary. This is what these macros do:

\def\gre@falsepenalty#1{}%
\def\gre@truepenalty#1{\penalty#1}%

\let\gre@hskip\hskip%
\let\gre@penalty\gre@truepenalty%
\xdef\gre@insidediscretionary{\number 0}%

% #1 is the type of discretionary, for penalty assignment. Recognized types:
%   - 0: clef change
\def\GreDiscretionary#1#2#3{%
  \global\let\gre@hskip\kern %
  \global\let\gre@penalty\gre@falsepenalty %
  \global\xdef\gre@insidediscretionary{\number 1}%
  \discretionary{%
    \global\gre@lastoflinecount=1\relax % (a good magic trick)
    \gre@debugmsg{bolshift}{discretionary pre lastoflinecount: \the\gre@lastoflinecount}%
    #2%
    }{%
    \global\gre@lastoflinecount=2\relax % (a good magic trick)
    \gre@debugmsg{bolshift}{discretionary post lastoflinecount: \the\gre@lastoflinecount}%
    }{%
    \gre@debugmsg{bolshift}{discretionary no lastoflinecount: \the\gre@lastoflinecount}%
    #3%
    }%
  \global\xdef\gre@insidediscretionary{\number 0}%
  \global\let\gre@hskip\hskip %
  \global\let\gre@penalty\gre@truepenalty %
  \relax %
}%

%%%%%%%%%%%%%%%%%%%%%%%%%%%%%%%%%%%%%%%%%%%%%%%%%%%%%%%%%%%%%%%%%%%%%%%%%%%%%%%%%%
%% macros for the typesetting of the clefs of the beginning of lines and custos
%%%%%%%%%%%%%%%%%%%%%%%%%%%%%%%%%%%%%%%%%%%%%%%%%%%%%%%%%%%%%%%%%%%%%%%%%%%%%%%%%%

% flag for showing the clef
\newif\ifgre@showclef%
\gre@showcleftrue

\def\gresetclef#1{%
  \IfStrEqCase{#1}{%
    {visible}%
      {\gre@showcleftrue}%
    {invisible}%
      {\gre@showcleffalse}%
    }[% all other cases
      \gre@error{Unrecognized option "#1" for \protect\gresetclef\MessageBreak Possible options are: 'visible' and 'invisible'}%
    ]%
}%

%% macro to define the clef that will appear at the beginning of the lines
%% #1 c or f: type of first clef
%% #2 int: line of first clef
%% #3 int: 0 if not space, 1 if normal space, 2 for short space
%% #4 int: height of the flat of first clef, 3 for no flat
%% #5, #6, #7 = #1, #2, #3 for second clef
\def\GreSetLinesClef#1#2#3#4#5#6#7{%
  \gre@save@additionalspaces %
  \directlua{gregoriotex.adjust_line_height(\gre@insidediscretionary, true)}%
  \gre@save@clef{#1}{#2}{#4}{#5}{#6}{#7}%
  \gre@localleftbox{%
    \gre@skip@temp@four = \gre@dimen@additionalleftspace\relax%
    \kern\gre@skip@temp@four %
    \copy\gre@box@lines% draws the lines
    \unkern %
    \ifgre@showclef%
      \gre@skip@temp@four = \gre@space@skip@afterclefnospace\relax%
      \hbox{\gre@typeclef{#1}{#2}{0}{#3}{#4}{#5}{#6}{#7}\hskip\gre@skip@temp@four}%
    \else %
      \gre@skip@temp@four = \gre@space@dimen@noclefspace\relax%
      \hbox{\kern\gre@skip@temp@four}%
    \fi %
  }%
  \xdef\gre@clefflatheight{#4}%
  \xdef\gre@cleftwoflatheight{#7}%
  \gre@restore@additionalspaces %
  \relax%
}%

% defines the largest clef of the score
% arguments are the same as \GreSetLinesClef except that the #3 of
% \GreSetLineClef is removed (always 1)
\def\GreSetLargestClef#1#2#3#4#5#6{%
  \gre@boxclef{#1}{#2}{0}{1}{#3}{#4}{#5}{#6}%
  \gre@update@clefwidth@largest{\wd\gre@box@temp@width}%
  \relax%
}%

\def\gre@save@clef#1#2#3#4#5#6{%
  \global\let\gre@clef=#1\relax%
  \xdef\gre@clefheight{#2}%
  \xdef\gre@clefflatheight{#3}%
  \global\let\gre@cleftwo=#4\relax%
  \xdef\gre@cleftwoheight{#5}%
  \xdef\gre@cleftwoflatheight{#6}%
}%

%% macro redrawing a key from clefnum, useful for vertical space changes
\def\gre@updatelinesclef{%
  \GreSetLinesClef{\gre@clef}{\gre@clefheight}{1}{\gre@clefflatheight}%
  {\gre@cleftwo}{\gre@cleftwoheight}{\gre@cleftwoflatheight}\relax %
}%

\newbox\gre@box@temp@clef%
\newbox\gre@box@temp@cleftwo%

% sets \gre@box@temp@width with the clef, the arguments are the same as \gre@typeclef
\def\gre@boxclef#1#2#3#4#5#6#7#8{%
  \global\setbox\gre@box@temp@width=\hbox{%
    \ifcase#7%
      \gre@typesingleclef{#1}{#2}{#3}{#5}%
    \else %
      \ifnum\numexpr (#7 - #2) * (#7 - #2) > 1 \relax %
        \setbox\gre@box@temp@clef=\hbox{\gre@typesingleclef{#1}{#2}{#3}{#5}}%
        \setbox\gre@box@temp@cleftwo=\hbox{\gre@typesingleclef{#6}{#7}{#3}{#8}}%
        \ifdim\wd\gre@box@temp@clef>\wd\gre@box@temp@cleftwo%
          \hbox to 0pt{\copy\gre@box@temp@cleftwo}\copy\gre@box@temp@clef%
        \else %
          \hbox to 0pt{\copy\gre@box@temp@clef}\copy\gre@box@temp@cleftwo%
        \fi %
      \else %
        \gre@typesingleclef{#1}{#2}{#3}{#5}%
        \gre@skip@temp@two=\gre@space@skip@clefflatspace\relax%
        \gre@hskip\gre@skip@temp@two %
        \gre@typesingleclef{#6}{#7}{#3}{#8}%
      \fi %
    \fi %
  }%
}

\def\gre@saveclefextrema#1#2{%
  % compute the clef extrema
  \ifcase#1\or % first@1
    \global\let\gre@pitch@cleftop\gre@pitch@e %
    \global\let\gre@pitch@clefbottom\gre@pitch@c %
    \ifcase#2\or\or % second@2
      \global\let\gre@pitch@cleftop\gre@pitch@g %
    \or % second@3
      \global\let\gre@pitch@cleftop\gre@pitch@i %
    \or % second@4
      \global\let\gre@pitch@cleftop\gre@pitch@k %
    \or % second@5
      \global\let\gre@pitch@cleftop\gre@pitch@m %
    \fi %
  \or % first@2
    \global\let\gre@pitch@cleftop\gre@pitch@g %
    \global\let\gre@pitch@clefbottom\gre@pitch@e %
    \ifcase#2\or % second@1
      \global\let\gre@pitch@clefbottom\gre@pitch@c %
    \or\or % second@3
      \global\let\gre@pitch@cleftop\gre@pitch@i %
    \or % second@4
      \global\let\gre@pitch@cleftop\gre@pitch@k %
    \or % second@5
      \global\let\gre@pitch@cleftop\gre@pitch@m %
    \fi %
  \or % first@3
    \global\let\gre@pitch@cleftop\gre@pitch@i %
    \global\let\gre@pitch@clefbottom\gre@pitch@g %
    \ifcase#2\or % second@1
      \global\let\gre@pitch@clefbottom\gre@pitch@c %
    \or % second@2
      \global\let\gre@pitch@clefbottom\gre@pitch@e %
    \or\or % second@4
      \global\let\gre@pitch@cleftop\gre@pitch@k %
    \or % second@5
      \global\let\gre@pitch@cleftop\gre@pitch@m %
    \fi %
  \or % first@4
    \global\let\gre@pitch@cleftop\gre@pitch@k %
    \global\let\gre@pitch@clefbottom\gre@pitch@i %
    \ifcase#2\or % second@1
      \global\let\gre@pitch@clefbottom\gre@pitch@c %
    \or % second@2
      \global\let\gre@pitch@clefbottom\gre@pitch@e %
    \or % second@3
      \global\let\gre@pitch@clefbottom\gre@pitch@g %
    \or\or % second@5
      \global\let\gre@pitch@cleftop\gre@pitch@m %
    \fi %
  \or % first@5
    \global\let\gre@pitch@cleftop\gre@pitch@m %
    \global\let\gre@pitch@clefbottom\gre@pitch@k %
    \ifcase#2\or % second@1
      \global\let\gre@pitch@clefbottom\gre@pitch@c %
    \or % second@2
      \global\let\gre@pitch@clefbottom\gre@pitch@e %
    \or % second@3
      \global\let\gre@pitch@clefbottom\gre@pitch@g %
    \or % second@4
      \global\let\gre@pitch@clefbottom\gre@pitch@i %
    \fi %
  \fi %
}%

\def\GreInitialClefPosition#1#2{%
  \ifgre@showclef %
    \gre@saveclefextrema{#1}{#2}%
  \fi %
}%

% macro that typesets the clef
% arguments are :
%% #1: the type of the clef : c or f
%% #2: the line of the clef (1 is the lowest)
%% #3: if we must use small clef characters (inside a line) or not 0: if not inside, 1 if inside
%% #4: 0: no space after, 1: normal space after, 2: short space after
%% #5: if 3, it means that we must not put a flat after the clef, otherwise it's the height of the flat
%% #6: the type of the secondary clef : c or f
%% #7: the line of the secondary clef (1 is the lowest)
%% #8: if 3, it means that we must not put a flat after the secondary clef, otherwise it's the height of the flat
\def\gre@typeclef#1#2#3#4#5#6#7#8{%
  \gre@saveclefextrema{#2}{#7}%
  \gre@boxclef{#1}{#2}{#3}{#4}{#5}{#6}{#7}{#8}%
  \ifcase#3%
    \gre@update@clefwidth@current{\wd\gre@box@temp@width}%
  \fi %
  \copy\gre@box@temp@width %
  \ifcase#4 %
    \gre@skip@temp@two=\gre@space@skip@afterclefnospace\relax%
  \or %
    \gre@skip@temp@two=\gre@space@skip@spaceafterlineclef\relax%
  \else %
    \gre@skip@temp@two=\gre@space@skip@shortspaceafterlineclef\relax%
  \fi %
  \gre@hskip\gre@skip@temp@two %
}%

% macro that typesets one clef
% arguments are :
%% #1: the type of the key : c or f
%% #2: the line of the key (1 is the lowest)
%% #3: if we must use small key characters (inside a line) or not 0: if not inside, 1 if inside
%% #4: if 3, it means that we must not put a flat after the key, otherwise it's the height of the flat
\def\gre@typesingleclef#1#2#3#4{%
  \ifcase#2 %
  \or%
    \gre@calculate@glyphraisevalue{\gre@pitch@c}{0}{}%
  \or%
    \gre@calculate@glyphraisevalue{\gre@pitch@e}{0}{}%
  \or%
    \gre@calculate@glyphraisevalue{\gre@pitch@g}{0}{}%
  \or%
    \gre@calculate@glyphraisevalue{\gre@pitch@i}{0}{}%
  \or%
    \gre@calculate@glyphraisevalue{\gre@pitch@k}{0}{}%
  \fi%
  \ifx c#1% we check if it is a c key
    \ifcase#3%
      \raise\gre@dimen@glyphraisevalue\hbox{\gre@fontchar@cclef}{}{}%
    \or%
      \raise\gre@dimen@glyphraisevalue\hbox{\gre@fontchar@incclef}{}{}%
    \fi%
  \else % we consider that it is a f key
    \ifcase#3%
      \raise\gre@dimen@glyphraisevalue\hbox{\gre@fontchar@fclef}{}{}%
    \or%
      \raise\gre@dimen@glyphraisevalue\hbox{\gre@fontchar@infclef}{}{}%
    \fi%
  \fi%
  \ifnum#4=3%
  \else %
    \gre@skip@temp@four = \gre@space@skip@clefflatspace\relax%
    \gre@hskip\gre@skip@temp@four %
    \GreFlat{#4}{1}{}{}{}%
  \fi %
  \relax %
}%

% macro that writes the initial key, and sets the next keys to the same value
% if #3 is 3, it means that we must not put a flat after the key, otherwise it's the height
% of the flat
\def\GreSetInitialClef#1#2#3#4#5#6#7{%
  \gre@save@clef{#1}{#2}{#3}{#4}{#5}{#6}%
  \ifgre@showclef%
    \ifnum\gre@initiallines=1\relax %
      \ifnum#7=1\relax %
        \gre@typeclef{#1}{#2}{0}{1}{#3}{#4}{#5}{#6}%
      \else %
        \gre@typeclef{#1}{#2}{0}{2}{#3}{#4}{#5}{#6}%
      \fi %
    \else %
      \gre@typeclef{#1}{#2}{0}{1}{#3}{#4}{#5}{#6}%
    \fi %
  \else%
    \gre@skip@temp@four = \gre@space@dimen@noclefspace\relax%
    \hbox{\kern\gre@skip@temp@four}%
  \fi %
  \GreSetLinesClef{#1}{#2}{1}{#3}{#4}{#5}{#6}%
  % if the initial is big, then we adjust the second line
  \ifnum\gre@biginitial=0\relax %
  \else %
    \gre@adjustsecondline %
  \fi %
  \relax%
}%

% macro called when the key changes
% #1 and #2 are the type and line of the clef
% #3 is 1 or 0 according to the need of a space before the clef. Useful for clefs after bars for example
% if #4 is 3, it means that we must not put a flat after the key, otherwise it's the height
% of the flat
\def\GreChangeClef#1#2#3#4#5#6#7{%
  % it makes no sense to change the clef when there is no clef...
  \gresetclef{visible}%
  \gre@save@clef{#1}{#2}{#4}{#5}{#6}{#7}%
  \ifnum\gre@insidediscretionary=0\relax %
    \GreSetLinesClef{#1}{#2}{1}{#4}{#5}{#6}{#7}%
  \fi %
  \ifnum#3=1\relax %
    \gre@skip@temp@four = \gre@space@skip@clefchangespace\relax%
    \gre@hskip\gre@skip@temp@four %
  \else %
    \ifdim\gre@skip@bar@lastskip=0pt\else % we're after a bar:
      % here it means that there is a bar before the clef, so we skip the difference between the normal space and the space around bars with clef changes
      \gre@skip@temp@four = -\gre@space@skip@spacearoundclefbars\relax%
      \gre@hskip\gre@skip@temp@four %
    \fi %
  \fi %
  \GreNoBreak %
  \gre@typeclef{#1}{#2}{1}{0}{#4}{#5}{#6}{#7}%
  \ifnum\gre@insidediscretionary=0\relax %
    \gre@skip@temp@four = \gre@space@skip@clefchangespace\relax%
  \else %
    \gre@skip@temp@four = \gre@space@skip@interwordspacenotes\relax%
  \fi %
  \gre@hskip\gre@skip@temp@four %
  \GreNoBreak %
  \relax%
}%

% custos just typesets a custos, useful for before the key changes for example
% #1 is the height
\def\GreCustos#1#2{%
  \GreNoBreak %
  \ifdim\gre@skip@bar@lastskip=0pt\else % we're after a bar:
    \kern-\gre@skip@bar@lastskip %
    \GreNoBreak %
  \fi %
  \ifgre@firstglyph% we check if it is the first glyph
    \global\gre@firstglyphfalse%
    \ifnum\gre@insidediscretionary=0\else %
      \gre@skip@temp@four=\gre@space@skip@interwordspacenotes %
      \kern\gre@skip@temp@four%
      \GreNoBreak %
    \fi %
  \else%
    \ifgre@endofscore %
      \gre@skip@temp@four=\gre@space@skip@spacebeforeeolcustos %
    \else %
      \gre@skip@temp@four=\gre@space@skip@spacebeforeinlinecustos %
    \fi %
    \kern\gre@skip@temp@four%
    \GreNoBreak %
  \fi %
  \gre@custosalteration{#1}{#2}%
  \GreNoBreak%
  \gre@calculate@glyphraisevalue{#1}{0}{}%
  %here we need some tricks to draw the line before the custos (for the color)
  \setbox\gre@box@temp@width=\hbox{\gre@pickcustos{#1}{1}}%
  \gre@dimen@temp@three=\wd\gre@box@temp@width %
  \ifgre@showlines %
    \ifnum#1<\gre@pitch@belowstaff\relax %
      \gre@additionalbottomcustoslinemiddle %
    \else\ifnum#1>\gre@pitch@abovestaff\relax %
      \gre@additionaltopcustoslinemiddle %
    \fi\fi %
  \fi %
  \raise \gre@dimen@glyphraisevalue%
  \copy\gre@box@temp@width %
  \ifdim\gre@skip@bar@lastskip=0pt\relax %
    % for now we consider we always have a bar after the custos
    % we don't want to end the line here
    \GreNoBreak %
    \gre@skip@temp@four = -\gre@space@skip@spacearoundclefbars\relax%
    \gre@hskip\gre@skip@temp@four %
    \GreNoBreak %
  \else %
    \GreNoBreak %
    \kern\gre@skip@bar@lastskip %
    \GreNoBreak %
  \fi %
  \directlua{gregoriotex.adjust_line_height(\gre@insidediscretionary)}%
  \relax %
}%

% typesets a custos for the end of the score
\def\GreFinalCustos#1#2{%
  \GreNoBreak%
  \GreCustos{#1}{#2}%
  \gre@endofglyphcommon %
}

% the argument is the height
\edef\gre@nextcustospitch{\gre@pitch@dummy}%
\edef\gre@nextcustosalteration{}%
\def\GreNextCustos#1#2{%
  \gre@debugmsg{custos}{nextcustos = #1,#2}%
  \ifnum\gre@insidediscretionary=0\relax %
    \gre@debugmsg{custos}{not discretionary}%
    \edef\gre@nextcustospitch{#1}%
    \edef\gre@nextcustosalteration{#2}%
    \ifgre@blockeolcustos\else%
      \gre@debugmsg{custos}{not blocked}%
      \gre@calculate@glyphraisevalue{#1}{0}{}%
      %here we need some tricks to draw the line before the custos (for the color)
      \setbox\gre@box@temp@width=\hbox{%
        % we type a hskip and the we type the custos
        \gre@hskip\gre@space@skip@spacebeforeeolcustos\relax %
        \gre@custosalteration{#1}{#2}%
        \GreNoBreak%
        \raise \gre@dimen@glyphraisevalue%
        \hbox{%
          \gre@pickcustos{#1}{2}\relax %
        }%
      }%
      \gre@dimen@temp@three=\wd\gre@box@temp@width %
      % we make \wd\gre@box@temp@sign contain the width of a custos
      \setbox\gre@box@temp@sign=\hbox{%
        \gre@custosalteration{#1}{#2}%
        \GreNoBreak%
        \gre@pickcustos{#1}{0}\relax %
      }%
      \gre@localrightbox{%
        \ifgre@showlines %
          \ifnum#1<\gre@pitch@belowstaff\relax %
            \gre@additionalbottomcustoslineend %
          \else\ifnum#1>\gre@pitch@abovestaff\relax %
            \gre@additionaltopcustoslineend %
          \fi\fi %
        \fi %
        \copy\gre@box@temp@width %
      }%
    \fi %
  \fi %
  \relax%
}%

% macro that typesets an additional line at the top for custos at end of line

\def\gre@additionaltopcustoslineend{%
  \gre@dimen@temp@five=\dimexpr(\gre@dimen@staffheight %
    + \gre@space@dimen@spacebeneathtext %
    + \gre@space@dimen@spacelinestext %
    + \gre@dimen@interstafflinespace %
    + \gre@dimen@additionalbottomspace %
    + \gre@dimen@currenttranslationheight)\relax %
  \raise\gre@dimen@temp@five %
  \hbox to 0pt{%
    \gre@style@additionalstafflines %
    \kern\gre@dimen@temp@three %
    \gre@dimen@temp@five=\dimexpr(\wd\gre@box@temp@sign + \gre@space@dimen@additionalcustoslineswidth)\relax%
    \kern-\gre@dimen@temp@five %
    \vrule width \gre@dimen@temp@five height \gre@dimen@stafflineheight\relax%
    \hss %
    \endgre@style@additionalstafflines%
  }%
  \relax %
}%

\def\gre@additionalbottomcustoslineend{%
  \gre@dimen@temp@five=\dimexpr(\gre@space@dimen@spacebeneathtext %
    + \gre@space@dimen@spacelinestext %
    + \gre@dimen@additionalbottomspace %
    + \gre@dimen@currenttranslationheight %
    - \gre@dimen@interstafflinespace %
    - \gre@dimen@stafflineheight)\relax %
  \raise\gre@dimen@temp@five %
  \hbox to 0pt{%
    \gre@style@additionalstafflines %
    \kern\gre@dimen@temp@three %
    \gre@dimen@temp@five=\dimexpr(\wd\gre@box@temp@sign+\gre@space@dimen@additionalcustoslineswidth)\relax%
    \kern-\gre@dimen@temp@five %
    \vrule width \gre@dimen@temp@five height \gre@dimen@stafflineheight\relax%
    \hss %
    \endgre@style@additionalstafflines%
  }%
  \relax %
}%

% same macros, but for a custos in the middle

\def\gre@additionaltopcustoslinemiddle{%
  \gre@dimen@temp@five=\dimexpr(\gre@dimen@staffheight %
    + \gre@space@dimen@spacebeneathtext %
    + \gre@space@dimen@spacelinestext %
    + \gre@dimen@interstafflinespace %
    + \gre@dimen@additionalbottomspace %
    + \gre@dimen@currenttranslationheight)\relax%
  \raise\gre@dimen@temp@five %
  \hbox to 0pt{%
    \gre@style@additionalstafflines %
    \hss %
    \kern\gre@dimen@temp@three %
    \gre@dimen@temp@five=\dimexpr((\gre@space@dimen@additionalcustoslineswidth*2)+\wd\gre@box@temp@sign)\relax %
    \vrule width \gre@dimen@temp@five height \gre@dimen@stafflineheight\relax%
    \hss %
    \endgre@style@additionalstafflines%
  }%
  \relax %
}%

\def\gre@additionalbottomcustoslinemiddle{%
  \gre@dimen@temp@five=\dimexpr(\gre@space@dimen@spacebeneathtext %
    + \gre@space@dimen@spacelinestext %
    + \gre@dimen@additionalbottomspace %
    + \gre@dimen@currenttranslationheight %
    - \gre@dimen@interstafflinespace %
    - \gre@dimen@stafflineheight)\relax%
  \raise\gre@dimen@temp@five %
  \hbox to 0pt{%
    \gre@style@additionalstafflines %
    \hss %
    \kern\gre@dimen@temp@three %
    \gre@dimen@temp@five=\dimexpr((\gre@space@dimen@additionalcustoslineswidth*2)+\wd\gre@box@temp@sign)\relax %
    \vrule width \gre@dimen@temp@five height \gre@dimen@stafflineheight\relax%
    \hss %
    \endgre@style@additionalstafflines%
  }%
  \relax %
}%

\newif\ifgre@usecustosalteration%
\gre@usecustosalterationtrue%
\def\gresetcustosalteration#1{%
  \IfStrEqCase{#1}{%
    {visible}{\gre@usecustosalterationtrue}%
    {invisible}{\gre@usecustosalterationfalse}%
  }[% all other cases
    \gre@error{Unrecognized option "#1" for \protect\gresetcustosalteration\MessageBreak Possible options are: 'visible' and 'invisible'}%
  ]%
}%
\def\gre@custosalteration#1#2{%
  \ifgre@usecustosalteration %
    \IfStrEq{#2}{}{}{\csname Gre#2\endcsname{#1}{0}{}{}{}}%
  \fi %
}%

% #2 is 0 for measurement only, 1 in the normal case, 2 for the right box
\def\gre@pickcustos#1#2{%
  % set attributes to adjust the line height for the pitch of the custos
  \ifcase#2% 0
  \or % 1
    \global\advance\gre@attr@glyph@id by 1\relax %
    \GreGlyphHeights{#1}{#1}%
  \or % 2
    \ifgre@eolshiftsenabled %
      \GreGlyphHeights{#1}{#1}%
    \fi %
  \fi %
  \ifcase#1%
  \or\or%
  \or\gre@fontchar@custostopmiddle %
  \or\gre@fontchar@custostoplong %
  \or\gre@fontchar@custostopshort %
  \or\gre@fontchar@custostoplong %
  \or\gre@fontchar@custostopshort %
  \or %
    \ifgre@haslinethree %
      \gre@fontchar@custostoplong %
    \else %
      \gre@fontchar@custosbottomlong %
    \fi %
  \or %
    \ifgre@haslinethree %
      \gre@fontchar@custostopshort %
    \else %
      \gre@fontchar@custosbottomshort %
    \fi %
  \or %
    \ifgre@haslinefour %
      \gre@fontchar@custostoplong %
    \else %
      \gre@fontchar@custosbottomlong %
    \fi %
  \or %
    \ifgre@haslinefour %
      \gre@fontchar@custostopshort %
    \else %
      \ifgre@haslinethree %
        \gre@fontchar@custosbottomshort %
      \else %
        \gre@fontchar@custosbottommiddle %
      \fi %
    \fi %
  \or %
    \ifgre@haslinefive %
      \gre@fontchar@custostoplong %
    \else %
      \gre@fontchar@custosbottomlong %
    \fi %
  \or %
    \ifgre@haslinefive %
      \gre@fontchar@custostopshort %
    \else %
      \ifgre@haslinefour %
        \gre@fontchar@custosbottomshort %
      \else %
        \gre@fontchar@custosbottommiddle %
      \fi %
    \fi %
  \or\gre@fontchar@custosbottomlong %
  \or %
    \ifgre@haslinefive %
      \gre@fontchar@custosbottomshort %
    \else %
      \gre@fontchar@custosbottommiddle %
    \fi %
  \or\gre@fontchar@custosbottomlong %
  \or\gre@fontchar@custosbottommiddle %
  \fi%
}%

%%%%%%%%%%%%%%%%%%%%%%%%%%%%%%%%%%%%%%%%%%%%%%%%%%%%%%%%%%%%%%%%%%%%%%%%%
%% macros for the typesetting of braces and other things above the score
%%%%%%%%%%%%%%%%%%%%%%%%%%%%%%%%%%%%%%%%%%%%%%%%%%%%%%%%%%%%%%%%%%%%%%%%%

\newif\ifgre@metapost@brace\gre@metapost@bracetrue %
\newif\ifgre@metapost@underbrace\gre@metapost@underbracetrue %
\newif\ifgre@metapost@curlybrace\gre@metapost@curlybracetrue %
\newif\ifgre@metapost@barbrace\gre@metapost@barbracetrue %
\def\gresetbracerendering{%
  \@ifnextchar[{\gre@setbracerendering}{\gre@setallbracerendering}%
}%
\def\gre@setallbracerendering#1{%
  \gre@@setbracerendering{brace}{#1}%
  \gre@@setbracerendering{underbrace}{#1}%
  \gre@@setbracerendering{curlybrace}{#1}%
  \gre@@setbracerendering{barbrace}{#1}%
}%
\def\gre@setbracerendering[#1]#2{%
  \IfStrEqCase{#1}{%
    {brace}{\gre@@setbracerendering{#1}{#2}}%
    {underbrace}{\gre@@setbracerendering{#1}{#2}}%
    {curlybrace}{\gre@@setbracerendering{#1}{#2}}%
    {barbrace}{\gre@@setbracerendering{#1}{#2}}%
  }[\gre@error{Unrecognized option "#1" for \protect\gresetbracerendering\MessageBreak Possible options are: 'brace', 'underbrace', 'curlybrace', and 'barbrace'}]%
}%
\def\gre@@setbracerendering#1#2{%
  \IfStrEqCase{#2}{%
    {font}{\csname gre@metapost@#1false\endcsname}%
    {metapost}{\csname gre@metapost@#1true\endcsname}%
  }[\gre@error{Unrecognized option "#2" for \protect\gresetbracerendering\MessageBreak Possible options are: 'font' and 'metapost'}]%
}%

\gdef\gre@fontchar@curlybrace{\gre@font@music\GreCPCurlyBrace}%
\gdef\gre@fontchar@brace{\gre@font@music\GreCPRoundBrace}%
\gdef\gre@fontchar@underbrace{\gre@font@music\GreCPRoundBraceDown}%

% the command to resize a box, \resizebox is provided by graphicx
\global\let\gre@resizebox\resizebox %

% #1: the width
% #2: a vertical shift
% #3: a horizontal shift
% #4: 1 if we shift to the beginning of the last glyph, 0 otherwise
% #5: 1 if we put an accentus above or not
\def\GreOverCurlyBrace#1#2#3#4#5{%
  \ifgre@boxing\else %
    \gre@brace@common{#1}{#2}{#3}{#4}{#5}{\gre@pitch@overbraceglyph}{\gre@fontchar@curlybrace}%
  \fi %
}%

% #1: the width
% #2: a vertical shift
% #3: a horizontal shift
% #4: 1 if we shift to the beginning of the last glyph, 0 otherwise
\def\GreOverBrace#1#2#3#4{%
  \ifgre@boxing\else %
    \gre@brace@common{#1}{#2}{#3}{#4}{0}{\gre@pitch@overbraceglyph}{\gre@fontchar@brace}%
  \fi %
}%

% #1: the width
% #2: a vertical shift
% #3: a horizontal shift
% #4: 1 if we shift to the beginning of the last glyph, 0 otherwise
\def\GreUnderBrace#1#2#3#4{%
  \ifgre@boxing\else %
    \gre@brace@common{#1}{#2}{#3}{#4}{0}{\gre@pitch@underbrace}{\gre@fontchar@underbrace}%
  \fi %
}%

% #1: the width
% #2: a vertical shift
% #3: a horizontal shift
% #4: 1 if we shift to the beginning of the last glyph, 0 otherwise
% #5: 1 if we put an accentus above, 0 if not
% #6: the pitch at which to compute the height
% #7: the brace character
\def\gre@brace@common#1#2#3#4#5#6#7{%
  \ifnum#4=1\relax %
    \setbox\gre@box@temp@sign=\hbox{\gre@fontchar@punctum}%
    \gre@dimen@temp@five=\wd\gre@box@temp@sign %
    \kern-\gre@dimen@temp@five %
  \fi %
  \gre@calculate@glyphraisevalue{#6}{13}{}%
  \advance\gre@dimen@glyphraisevalue by #2\relax %
  \hbox to 0pt{%
    \gre@skip@temp@four = #3\relax %
    \kern\gre@skip@temp@four %
    \raise\gre@dimen@glyphraisevalue\hbox{%
      \ifx#7\gre@fontchar@curlybrace %
        \ifgre@metapost@curlybrace %
          \gre@draw@curlybrace{#1}%
        \else %
          \gre@draw@fontbrace{#1}{#7}%
        \fi %
      \else %
        \ifx#7\gre@fontchar@brace %
          \ifgre@metapost@brace %
            \gre@draw@brace{#1}%
          \else %
            \gre@draw@fontbrace{#1}{#7}%
          \fi %
        \else %
          \ifx#7\gre@fontchar@underbrace %
            \ifgre@metapost@underbrace %
              \gre@draw@underbrace{#1}%
            \else %
              \gre@draw@fontbrace{#1}{#7}%
            \fi %
          \fi %
        \fi %
      \fi %
    }%
    \hss %
    \ifnum#5=1\relax %
      \gre@calculate@glyphraisevalue{\gre@pitch@overbrace}{13}{}%
      \advance\gre@dimen@glyphraisevalue by \gre@space@dimen@curlybraceaccentusshift\relax%
      \raise\gre@dimen@glyphraisevalue\hbox{%
        \gre@font@music\GreCPAccentus\relax %
      }%
      \hss %
    \fi %
  }%
  \ifnum#4=1\relax %
    \kern\gre@dimen@temp@five %
  \fi %
  \relax %
}%
% #1 : the width
% #2 : the brace character
\def\gre@draw@fontbrace#1#2{%
  \setbox\gre@box@temp@sign=\hbox{#2}%
  \gre@resizebox{#1}{\ht\gre@box@temp@sign}{#2}%
}%

% #1: the width, or * for the bar brace width
% this can't have @ in the name because of the metapost catcode settings
\def\grebracemetapostpreamble#1{%
  % multiplier to convert gre@factor to em-size in bp
  % (100000 sp) * (72 bp/in) / (65536 sp/pt) / 72.27 (pt/in)
  factor = 1.5201782378;
  scale = \the\gre@factor * factor;
  % assuming #1 is in bp, we convert it back into the internal unit of the
  % calculation, ems
  width = \directlua{gregoriotex.width_to_bp([[#1]], "\grebarbracewidth * scale")} / scale;
}%

% #1: the width
\def\gre@draw@curlybrace#1{%
  \gre@debugmsg{general}{curly brace width = #1}%
  \gre@metapost{
    \grebracemetapostpreamble{#1}
    transform t;
    t = identity scaled scale;
    p := (width - 0.1416) / 2;
    q := (width - 0.08626) / 2;
    if (p < 0.1005) or (q < 0.12817):
      t := t xscaled (width / 0.34261);
      p := 0.1005;
      q := 0.12817;
    fi;
    f := 1 + (p / 4);
    beginfig(0);
    fill
      (  (0,0)
      -- (0.00651,0)
      .. controls (((0.03239**f)*p)+0.00651,0.01763) and (((0.08502**f)*p)+0.00651,0.03065) .. (((0.15789**f)*p)+0.00651,0.03906)
      .. controls (((0.23347**f)*p)+0.00651,0.04801) and (((0.32659**f)*p)+0.00651,0.05249) .. (((0.43725**f)*p)+0.00651,0.05249)
      .. controls (((0.49663**f)*p)+0.00651,0.05249) and (((0.59109**f)*p)+0.00651,0.05086) .. (((0.72065**f)*p)+0.00651,0.04761)
      .. controls (((0.82861**f)*p)+0.00651,0.04408) and (((0.92173**f)*p)+0.00651,0.04232) .. (p+0.00651,0.04232)
      .. controls (p+0.0198,0.04232) and (p+0.03242,0.04842) .. (p+0.04435,0.06063)
      .. controls (p+0.05602,0.07365) and (p+0.06483,0.0906) .. (p+0.0708,0.11149)
      .. controls (p+0.07677,0.0906) and (p+0.08558,0.07365) .. (p+0.09725,0.06063)
      .. controls (p+0.10918,0.04842) and (p+0.1218,0.04232) .. (p+0.13509,0.04232)
      .. controls (p+((1-(0.92173**f))*p)+0.13509,0.04232) and (p+((1-(0.82861**f))*p)+0.13509,0.04408) .. (p+((1-(0.72065**f))*p)+0.13509,0.04761)
      .. controls (p+((1-(0.59109**f))*p)+0.13509,0.05086) and (p+((1-(0.49663**f))*p)+0.13509,0.05249) .. (p+((1-(0.43725**f))*p)+0.13509,0.05249)
      .. controls (p+((1-(0.32659**f))*p)+0.13509,0.05249) and (p+((1-(0.23347**f))*p)+0.13509,0.04801) .. (p+((1-(0.15789**f))*p)+0.13509,0.03906)
      .. controls (p+((1-(0.08502**f))*p)+0.13509,0.03065) and (p+((1-(0.03239**f))*p)+0.13509,0.01763) .. (p+p+0.13509,0)
      -- (p+p+0.13509,0) -- (q+q+0.08626,0) -- (q+q+0.08626,0)
      .. controls (q+((1-(0.02117**f))*q)+0.08626,0.02658) and (q+((1-(0.08466**f))*q)+0.08626,0.04774) .. (q+((1-(0.19048**f))*q)+0.08626,0.06348)
      .. controls (q+((1-(0.29418**f))*q)+0.08626,0.07975) and (q+((1-(0.42434**f))*q)+0.08626,0.08789) .. (q+((1-(0.58095**f))*q)+0.08626,0.08789)
      .. controls (q+((1-(0.63386**f))*q)+0.08626,0.08789) and (q+((1-(0.70159**f))*q)+0.08626,0.08626) .. (q+((1-(0.78413**f))*q)+0.08626,0.08301)
      .. controls (q+((1-(0.87937**f))*q)+0.08626,0.07894) and (q+((1-(0.95132**f))*q)+0.08626,0.0769) .. (q+0.08626,0.0769)
      .. controls (q+0.06219,0.0769) and (q+0.05013,0.10644) .. (q+0.04679,0.13102)
      -- (q+0.04679,0.13102) -- (q+0.03947,0.13102) -- (q+0.03947,0.13102)
      .. controls (q+0.03614,0.10644) and (q+0.02407,0.0769) .. (q+0,0.0769)
      .. controls (((0.95132**f)*q)+0,0.0769) and (((0.87937**f)*q)+0,0.07894) .. (((0.78413**f)*q)+0,0.08301)
      .. controls (((0.70159**f)*q)+0,0.08626) and (((0.63386**f)*q)+0,0.08789) .. (((0.58095**f)*q)+0,0.08789)
      .. controls (((0.42434**f)*q)+0,0.08789) and (((0.29418**f)*q)+0,0.07975) .. (((0.19048**f)*q)+0,0.06348)
      .. controls (((0.08466**f)*q)+0,0.04774) and (((0.02117**f)*q)+0,0.02658) .. (0,0)
      -- cycle
      ) transformed t;
    setbounds currentpicture to
      ( (0,-0.67980) -- (q+q+0.08626,-0.67980) -- (q+q+0.08626,0.13102) -- (0,0.13102) -- cycle )
      transformed t;
    endfig;
  }%
}%

% #1: the width
\def\gre@draw@brace#1{%
  \gre@draw@roundbrace{#1}{.89500}{
       (-.00192,.70347) % start the top
    .. controls (-.00192,.70525) and (-.00134,.70740)
    .. (.00000,.71000) -- (.00000,.71000){curl c}
    .. (p/2,.89500) % top of the center
    .. {curl c}(p,.71000) -- (p,.71000)
    .. controls (p+.00134,.70740) and (p+.00192,.70525)
    .. (p+.00192,.70347) % end the top
    .. controls (p+.00192,.69508) and (p-.01103,.69500)
    .. (p-.01856,.69500) % start the bottom
    .. controls (p-.01856,.69500) and (p-.01900,.69500)
    .. (p-.01900,.69500) -- (p-.01900,.69500){curl c}
    .. (p/2,.86400) % bottom of the center
    .. {curl c}(.01900,.69500) -- (.01900,.69500)
    .. controls (.01900,.69500) and (.01856,.69500)
    .. (.01856,.69500) % end the bottom
    .. controls (.01103,.69500) and (-.00192,.69508)
    .. (-.00192,.70347) -- cycle
  }%
}%

% #1: the width
\def\gre@draw@underbrace#1{%
  \gre@draw@roundbrace{#1}{.25500}{
       (-.00192,.24653) % start the bottom
    .. controls (-.00192,.24475) and (-.00134,.24260)
    .. (.00000,.24000) -- (.00000,.24000){curl c}
    .. (p/2,.05500) % bottom of the center
    .. {curl c}(p,.24000) -- (p,.24000)
    .. controls (p+.00134,.24260) and (p+.00192,.24475)
    .. (p+.00192,.24653) % end the bottom
    .. controls (p+.00192,.25492) and (p-.01103,.25500)
    .. (p-.01856,.25500) % start the top
    .. controls (p-.01856,.25500) and (p-.01900,.25500)
    .. (p-.01900,.25500) -- (p-.01900,.25500){curl c}
    .. (p/2,.08600) % top to the center
    .. {curl c}(.01900,.25500) -- (.01900,.25500)
    .. controls (.01900,.25500) and (.01856,.25500)
    .. (.01856,.25500) % end the top
    .. controls (.01103,.25500) and (-.00192,.25492)
    .. (-.00192,.24653) -- cycle
  }%
}%

% #1: the width
% #2: height of the bounding box
% #3: metapost commands to draw the outline
\def\gre@draw@roundbrace#1#2#3{%
  \gre@debugmsg{general}{round brace width = #1}%
  \gre@metapost{
    \grebracemetapostpreamble{#1}
    p = width;
    transform t;
    t = identity scaled scale;
    c = 1;
    if p < 0.200:
      c := 0;
      t := t xscaled (p/0.200);
      p := 0.200;
    elseif p < 0.350:
      c := (p - 0.200) / 0.350;
    fi;
    beginfig(0);
    fill ( #3 ) transformed t;
    setbounds currentpicture to
      ( (0,0) -- (p,0) -- (p,#2) -- (0,#2) -- cycle )
      transformed t;
    endfig;
  }%
}%

% #1 : x-distance
% #2 : y-distance
% #3 : -1 for below, 1 for above
\def\gre@draw@slur#1#2#3{%
  \gre@metapost{
    \grebracemetapostpreamble{\directlua{gregoriotex.hypotenuse([[#1]],[[#2]])}}
    transform t;
    t = identity scaled scale;
    t := t xscaled width;
    t := t rotated \directlua{gregoriotex.rotation([[#1]],[[#2]])};
    beginfig(0);
    fill
      (  (0,0)
      .. controls (0.3,#3*0.20) and (0.7,#3*0.20)
      .. (1,0)
      -- (1,0)
      .. controls (0.7,#3*0.22) and (0.3,#3*0.22)
      .. (0,0)
      -- cycle
      ) transformed t withpen pencircle scaled 0.3;
    endfig;
  }%
}%

\newdimen\greslurheight
% #1 : height
% #2 : -1 for below, 1 for above
% #3 : 0 = no left-shift, 1 = left-shift 1 punctum, 2 = left shift 1/2 punctum
% #4 : x-distance
% #5 : y-distance
% #6 : end height if #6 not given
\def\GreSlur#1#2#3#4#5#6{%
  \ifgre@boxing\else %
    \ifnum\number#2 > 0\relax %
      \gre@calculate@glyphraisevalue{#1}{15}{}%
    \else %
      \gre@calculate@glyphraisevalue{#1}{16}{}%
    \fi %
    \hbox to 0pt{\raise\gre@dimen@glyphraisevalue\hbox{%
      \ifcase#3 % 0
      \or % 1
        \setbox\gre@box@temp@sign=\hbox{\gre@fontchar@punctum}%
        \kern-\wd\gre@box@temp@sign %
      \or % 2
        \setbox\gre@box@temp@sign=\hbox{\gre@fontchar@punctum}%
        \kern-.5\wd\gre@box@temp@sign %
      \fi %
      \if\relax\detokenize{#5}\relax %
        \gre@dimen@temp@five=\gre@dimen@glyphraisevalue\relax %
        \ifnum\number#2 > 0\relax %
          \gre@calculate@glyphraisevalue{#6}{15}{}%
        \else %
          \gre@calculate@glyphraisevalue{#6}{16}{}%
        \fi %
        \greslurheight = \dimexpr(\gre@dimen@glyphraisevalue - \gre@dimen@temp@five)\relax %
      \else %
        \greslurheight = #5\relax %
      \fi %
      \gre@draw@slur{#4}{\the\greslurheight}{#2}%
    }}%
  \fi %
  \relax %
}%

% #1: id of the variable length brace within the score
\def\GreVarBraceLength#1{%
  \ifgre@boxing\else %
    \directlua{gregoriotex.var_brace_len(#1)}%
  \fi %
}%

% #1: id of the variable length brace within the score
% #2: 0 = no left-shift, 1 = left-shift 1 punctum, 2 = left shift 1/2 punctum
% #3: 1 if the start, 2 if the end
\def\GreVarBraceSavePos#1#2#3{%
  \ifgre@boxing\else %
    \ifcase#2 % 0
      \gre@debugmsg{general}{save case 0}%
      \gre@savepos %
    \or % 1
      \setbox\gre@box@temp@sign=\hbox{\gre@fontchar@punctum}%
      \kern-\wd\gre@box@temp@sign %
      \gre@savepos %
      \kern\wd\gre@box@temp@sign %
    \or % 2
      \gre@debugmsg{general}{save case 2}%
      \setbox\gre@box@temp@sign=\hbox{\gre@fontchar@punctum}%
      \gre@dimen@temp@five=\wd\gre@box@temp@sign %
      \kern-.5\wd\gre@box@temp@sign %
      \gre@savepos %
      \kern.5\wd\gre@box@temp@sign %
    \fi %
    \directlua{gregoriotex.save_length(#1, #3)}%
  \fi %
  \relax %
}%

%%%%%%%%%%%%%%%%%%%%%%%%%%%%%%%%%%%%%%%%%%%%%%%%%%%%%%%%%%%%%%%%%%%%%%%%%%%%%%
%% macros for the typesetting of punctum mora, auctum duplex and choral signs
%%%%%%%%%%%%%%%%%%%%%%%%%%%%%%%%%%%%%%%%%%%%%%%%%%%%%%%%%%%%%%%%%%%%%%%%%%%%%%

% a function to typeset a punctum mora,
% #1 is the letter of the height of the punctum mora
% #2 is
%   - 0 in the general case
%   - 1 if we make the punctum mora 0-width (e.g!hv)
%   - 2 if we shift the width of one punctum to the left (g.e)
%   - 3 same as 2 but with ambitus of one (g.f)
% #3 is 1 in case of a punctommora in the note before the last note of a podatus, porrectus or torculus resupinus, 0 otherwise.
% #4 is 1 if we are at a punctum inclinatum, 0 otherwise
\def\gre@punctum@mora#1#2#3#4{%
  \GreNoBreak %
  \ifcase#2\relax %
    \gre@skip@temp@four = \gre@space@skip@spacebeforesigns\relax%
    \hskip\gre@skip@temp@four%
  \or %
    \gre@skip@temp@four = \gre@space@skip@spacebeforesigns\relax%
    \kern\gre@skip@temp@four %
  \or %
    % to get the widht of a punctum minus a line, we calculate the width of a flexus (with ambitus of two) minus the width of a punctum
    \setbox\gre@box@temp@width=\hbox{\gre@font@music\GreCPPesQuadratumLongqueueThreeNothing}%
    \gre@dimen@temp@five=\wd\gre@box@temp@width %
    \setbox\gre@box@temp@width=\hbox{\gre@font@music\GreCPPunctum}%
    \advance\gre@dimen@temp@five by -\wd\gre@box@temp@width %
    \kern-\gre@dimen@temp@five %
    \gre@skip@temp@four = \gre@space@skip@spacebeforesigns\relax%
    \kern\gre@skip@temp@four %
  \or %
    \setbox\gre@box@temp@width=\hbox{\gre@font@music\GreCPPunctum}%
    \gre@dimen@temp@five=\wd\gre@box@temp@width %
    \kern-\gre@dimen@temp@five %
    \gre@skip@temp@four = \gre@space@skip@spacebeforesigns\relax%
    \kern\gre@skip@temp@four %
  \fi %
  \ifnum#2=0\relax %
    \global\gre@lastendswithmoratrue%
  \fi %
  \ifcase#3\relax % 0
    \gre@calculate@glyphraisevalue{#1}{4}{}%
  \or% 1
    \gre@calculate@glyphraisevalue{#1}{8}{}%
  \or% 2
    \gre@calculate@glyphraisevalue{#1}{14}{}%
  \fi %
  % here we shift a bit left in the case where we have a punctum inclinatum on a line
  \ifnum#4=1\relax %
    \ifgre@isonaline %
      \gre@dimen@temp@three=\dimexpr(3700sp * \gre@factor)\relax %
      \kern-\gre@dimen@temp@three %
      \gre@dimen@temp@three=\dimexpr(4500sp * \gre@factor)\relax%
      \advance\gre@dimen@glyphraisevalue by -\gre@dimen@temp@three %
    \else %
      \gre@dimen@temp@three=\dimexpr(2500sp * \gre@factor)\relax %
      \advance\gre@dimen@glyphraisevalue by -\gre@dimen@temp@three %
    \fi %
  \fi %
  \GreNoBreak %
  \raise \gre@dimen@glyphraisevalue \hbox{\gre@fontchar@punctummora}%
  \GreNoBreak %
  \ifcase#2\relax\or %
    \setbox\gre@box@temp@width=\hbox{\gre@fontchar@punctummora}%
    \gre@skip@temp@four = -\wd\gre@box@temp@width %
    \kern\gre@skip@temp@four%
    \gre@skip@temp@four = -\gre@space@skip@spacebeforesigns\relax%
    \kern\gre@skip@temp@four %
  \or %
    \setbox\gre@box@temp@width=\hbox{\gre@fontchar@punctummora}%
    \gre@skip@temp@four = -\wd\gre@box@temp@width %
    \kern\gre@skip@temp@four%
    \gre@skip@temp@four = -\gre@space@skip@spacebeforesigns\relax%
    \kern\gre@skip@temp@four %
    \kern\gre@dimen@temp@five %
  \or %
    \setbox\gre@box@temp@width=\hbox{\gre@fontchar@punctummora}%
    \gre@skip@temp@four = -\wd\gre@box@temp@width %
    \kern\gre@skip@temp@four%
    \gre@skip@temp@four = -\gre@space@skip@spacebeforesigns\relax%
    \kern\gre@skip@temp@four %
    \kern\gre@dimen@temp@five %
  \fi %
  \GreNoBreak %
  \relax%
}%

\def\GrePunctumMora#1#2#3#4{%
  \ifgre@disablemora\else %
    \gre@punctum@mora{#1}{#2}{#3}{#4}%
  \fi %
  \relax %
}

% a function to typeset an augmentum duplex, easy enough to be understood...
\def\GreAugmentumDuplex#1#2#3{%
  \ifgre@boxing\else %
    \GrePunctumMora{#1}{1}{#3}{0}%
  \fi %
  \GrePunctumMora{#2}{0}{0}{0}%
  \relax %
}%

% quite simple function: #1 is the height, #2 is the string, #3 is #2 of punctum mora, #4 is #3 of punctum mora
% #3 is 1 if it must be a bit higher
\def\GreLowChoralSign#1#2#3{%
  \GreNoBreak %
  \gre@skip@temp@four = \gre@space@skip@beforelowchoralsignspace\relax%
  \hskip\gre@skip@temp@four %
  \GreNoBreak %
  \ifnum#3=1\relax %
    \gre@calculate@glyphraisevalue{#1}{12}{}%
  \else %
    \gre@calculate@glyphraisevalue{#1}{10}{}%
  \fi %
  \raise\gre@dimen@glyphraisevalue\hbox{\gre@style@lowchoralsign#2\endgre@style@lowchoralsign}%
  \relax %
}%

\def\GreHighChoralSign#1#2#3{%
  \GreNoBreak %
  \gre@vepisemaorrare{#1}{#3}{}{3}{\gre@style@highchoralsign#2\endgre@style@highchoralsign}%
  \relax %
}%

%%%%%%%%%%%%%%%%%%%%%%%%%%%%%%%%%%%%%%%%%%%%%%%%%%%
%% macros for the typesetting of vertical episema
%%%%%%%%%%%%%%%%%%%%%%%%%%%%%%%%%%%%%%%%%%%%%%%%%%%

\newbox\gre@box@temp@sign%

% a macro to help typesetting vertical episema.
% #1 is an offset glyph (see #3 below)
% #2 represents the glyph upon which the sign is to be centered
% #3 is a case number
%    0 : go back to the beginning of the previous glyph and then forward half
%        the width of #2; this puts the sign at the beginning of the previous
%        glyph, whose first note is the size of #2
%    1 : go back half the width of #2; this puts the sign at the end of the
%        previous glyph, whose last note is the size of #2
%    2 : go back the width of #1 and then foward half the width of #2; this
%        puts the sign at the glyph from the end that starts at #1's width from
%        the end
%    3 : go back to the beginning of the previous glyph and then forward the
%        width of #1 and then back half the width of #2; this puts the sign at
%        the glyph from the start that ends at #1's width from the start
% #4 is a shift that we want to get applied, useful for punctum inclinatum for example
% #5 is the glyph number.
% #6 is the type of sign (1: vertical episema, 2: rare sign, 3: choral sign)
% #7 is the choral sign if relevant
\def\gre@vepisemaorrareaux#1#2#3#4#5#6#7{%
  % first we set \gre@dimen@temp@three to the width of the last glyph
  \gre@dimen@temp@three=\gre@dimen@lastglyphwidth\relax%
  \setbox\gre@box@temp@sign=\hbox{\gre@font@music #2}%
  \gre@dimen@temp@two=\dimexpr(\wd\gre@box@temp@sign / 2)\relax %
  \ifcase#3%
  % tempwidth is the width of the last glyph
    \advance\gre@dimen@temp@three by -\gre@dimen@temp@two %
  \or%
    \gre@dimen@temp@three=\gre@dimen@temp@two %
  \or%
    \setbox\gre@box@temp@sign=\hbox{\gre@font@music #1}%
    \gre@dimen@temp@three=\dimexpr(\wd\gre@box@temp@sign - \gre@dimen@temp@two)\relax %
  \or %
    \setbox\gre@box@temp@sign=\hbox{\gre@font@music #1}%
    \advance\gre@dimen@temp@three by \dimexpr(-\wd\gre@box@temp@sign + \gre@dimen@temp@two)\relax %
  \fi%
  \kern-\gre@dimen@temp@three % we do it here because of the now-removed ictus (chironomy)
  % then we draw the sign
  \ifcase#6\or %
    % vertical episema
    \setbox\gre@box@temp@sign=\hbox{\gre@fontchar@verticalepisema}%
  \or % rare sign
    \setbox\gre@box@temp@sign=\hbox{\gre@font@music#5}%
  \or % choral sign
    \setbox\gre@box@temp@sign=\hbox{#7}%
  \or % brace above bar
    \setbox\gre@box@temp@sign=\hbox{\gre@render@barbrace}%
  \fi %
  % we set tempwidth to half a punctum malus half the sign width, so that the centers are aligned
  \gre@dimen@temp@two=\dimexpr(\wd\gre@box@temp@sign / 2)\relax %
  \advance\gre@dimen@temp@three by \gre@dimen@temp@two %
  \kern-\gre@dimen@temp@two%
  \gre@skip@temp@four = #4sp%
  \kern \gre@skip@temp@four%
  \raise \gre@dimen@glyphraisevalue \copy\gre@box@temp@sign %
  \kern -\gre@skip@temp@four%
  % and finally we go back to the end of the glyph, where we were first
  \advance\gre@dimen@temp@three by -2\gre@dimen@temp@two %
  \kern\gre@dimen@temp@three%
  \relax%
}%

\directlua{gregoriotex.emit_offset_macros()}%

% a function to typeset a vertical episema or a rare accent (like accentus,
% circulus, etc.).  This function must be called after a call to \GreGlyph.
% #1 is the letter of the height of the episema (not the height of the note
%    it corresponds to.
% #2 is note position case as in the table above
% #3 is the sign glyph
% #4 is type (1: vertical episema, 2: rare sign, 3: choral sign, 4: brace above the bar)
% #5 is the choral sign if relevant
\def\gre@vepisemaorrare#1#2#3#4#5{%
  \ifgre@boxing\else %
    \ifcase#4\or %
      % if it is a vertical episema, we call the normal calculateglyphvalue
      \gre@calculate@glyphraisevalue{#1}{3}{}%
    \or %
      % if it is not, we call it with 6 as second argument, it will give us the height of the rare signs (accentus, etc.) the first argument is m if the pitch is < k, otherwise it's n.
      \ifnum#1<\gre@pitch@raresign\relax %
        \gre@calculate@glyphraisevalue{\gre@pitch@raresign}{6}{}%
      \else %
        {%
          \gre@count@temp@three=\numexpr(#1 + 1)\relax %
          \gre@calculate@glyphraisevalue{\gre@count@temp@three}{6}{}%
        }%
      \fi %
    \or % if it's a choral sign
      \gre@calculate@glyphraisevalue{#1}{11}{}%
    \or % if it's the brace above the bar
      \gre@calculate@glyphraisevalue{#1}{13}{}%
    \fi %
    \gre@v@case{#2}{#3}{#4}{#5}%
  \fi %
  \relax%
}%

\def\GreVEpisema#1#2{%
  \ifgre@disablevepisema\else %
    \gre@vepisemaorrare{#1}{#2}{\GreCPVEpisema}{1}{}%
  \fi %
  \relax %
}%

\def\GreBarBrace#1{%
  \gre@vepisemaorrare{\gre@pitch@overbraceglyph}{#1}{%
    \gre@render@barbrace%
  }{4}{}%
  \relax %
}%
\def\gre@render@barbrace{%
  \hbox{%
    \ifgre@metapost@barbrace %
      \gre@draw@brace{*}%
    \else %
      \gre@fontchar@abovebarbrace %
    \fi %
  }%
}%

\def\GreBarVEpisema#1{%
  \gre@vepisemaorrare{\gre@pitch@barvepisema}{#1}{\GreCPVEpisema}{1}{}%
  \relax %
}%

\def\GreAccentus#1#2{%
  \gre@vepisemaorrare{#1}{#2}{\GreCPAccentus}{2}{}%
  \relax %
}%

\def\GreSemicirculus#1#2{%
  \gre@vepisemaorrare{#1}{#2}{\GreCPSemicirculus}{2}{}%
  \relax %
}%

\def\GreCirculus#1#2{%
  \gre@vepisemaorrare{#1}{#2}{\GreCPCirculus}{2}{}%
  \relax %
}%

\def\GreReversedAccentus#1#2{%
  \gre@vepisemaorrare{#1}{#2}{\GreCPAccentusReversus}{2}{}%
  \relax %
}%

\def\GreReversedSemicirculus#1#2{%
  \gre@vepisemaorrare{#1}{#2}{\GreCPSemicirculusReversus}{2}{}%
  \relax %
}%

\def\GreMusicaFictaFlat#1#2{%
  \gre@vepisemaorrare{#1}{#2}{\GreCPFlat}{2}{}%
  \relax %
}%

\def\GreMusicaFictaNatural#1#2{%
  \gre@vepisemaorrare{#1}{#2}{\GreCPNatural}{2}{}%
  \relax %
}%

\def\GreMusicaFictaSharp#1#2{%
  \gre@vepisemaorrare{#1}{#2}{\GreCPSharp}{2}{}%
  \relax %
}%

%%%%%%%%%%%%%%%%%%%%%%%%%%%%%%%%%%%%%%%%%%%%%%%%%%
%% macros for the typesetting horizontal episema
%%%%%%%%%%%%%%%%%%%%%%%%%%%%%%%%%%%%%%%%%%%%%%%%%%


% a macro that will help in the typesetting of a horizontal episema and additional lines,
% #1 is an offset glyph (see #3, below)
% #2 is the episema glyph
% #3 is a case number, similar in nature to #3 in gre@vepisemaorrareaux
%    0 : go back to the beginning of the previous glyph; this starts the
%        episema at the beginning of the previous glyph
%    1 : stay at the end of the glyph; doesn't make much sense to use this
%    2 : go back the width of #1; this starts the episema at the glyph from
%        the end that starts at #1's width from the end
%    3 : go back to the beginning of the previous glyph and then forward the
%        width of #1; this starts the episema at the glyph from the start that
%        starts just after #1's width from the start
%    4 : go back to the beginning of the previous glyph and then forward the
%        width of #1, then back the width of #2; this ends the episema at the
%        end of #1
% #4 argument is the same as in hepisorline
\def\gre@hepisorlineaux#1#2#3#4{%
  \ifcase#3% case 0
  % first we set \gre@dimen@temp@three to the width of the last glyph
    \gre@dimen@temp@three=\gre@dimen@lastglyphwidth\relax%
  \or % case 1
    \gre@dimen@temp@three=0 pt\relax %
  \or % case 2
    \setbox\gre@box@temp@sign=\hbox{\gre@font@music #1}%
    \gre@dimen@temp@three=\wd\gre@box@temp@sign%
  \or % case 3
    \setbox\gre@box@temp@sign=\hbox{\gre@font@music #1}%
    \gre@dimen@temp@three=\dimexpr(\gre@dimen@lastglyphwidth - \wd\gre@box@temp@sign)\relax %
  \or % case 4
    \setbox\gre@box@temp@sign=\hbox{\gre@font@music #1}%
    \gre@dimen@temp@three=\dimexpr(\gre@dimen@lastglyphwidth - \wd\gre@box@temp@sign)\relax %
    \setbox\gre@box@temp@sign=\hbox{\gre@font@music #2}%
    \advance\gre@dimen@temp@three by \wd\gre@box@temp@sign %
  \fi%
  \kern-\gre@dimen@temp@three %
  % then we draw the sign, and go back to the beginning of the sign
  \setbox\gre@box@temp@sign=\hbox{\gre@font@music#2}%
  % we set tempwidth to half a punctum malus half the sign width, so that the centers are aligned
  \gre@dimen@temp@two=\wd\gre@box@temp@sign %
  \ifcase#4%
    %case of hepisema
    \raise \gre@dimen@glyphraisevalue \copy\gre@box@temp@sign %
  \or %
    %case of hepisema at the bottom
    \raise \gre@dimen@glyphraisevalue \copy\gre@box@temp@sign %
  \or % case of a line at the top
    \ifnum\gre@count@lastglyphiscavum=2\relax %
      \gre@drawadditionalline{0}{\gre@dimen@temp@two}{0}{}{1}{}%
    \else %
      \gre@drawadditionalline{0}{\gre@dimen@temp@two}{1}{}{1}{}%
    \fi %
    \gre@dimen@temp@two=0pt\relax %
  \or % case of a line at the bottom
    \ifnum\gre@count@lastglyphiscavum=2\relax %
      \gre@drawadditionalline{1}{\gre@dimen@temp@two}{0}{}{1}{}%
    \else %
      \gre@drawadditionalline{1}{\gre@dimen@temp@two}{1}{}{1}{}%
    \fi %
    \gre@dimen@temp@two=0pt\relax %
  \or %
    %case of choral sign
    \raise \gre@dimen@glyphraisevalue \copy\gre@box@temp@sign %
  \or %
  \fi %
  % and finally we go back to the end of the glyph, where we were first
  \advance\gre@dimen@temp@three by -\gre@dimen@temp@two %
  \kern\gre@dimen@temp@three %
  \relax%
}%

% another dumb top function
\def\GreAdditionalLine#1#2#3{%
  \ifgre@showlines %
    \ifgre@boxing\else %
      \xdef\gre@dimen@savedglyphraise{\the\gre@dimen@glyphraisevalue}%
      \gre@style@additionalstafflines %
      \gre@hepisorline{\gre@pitch@dummy}{#1}{#2}{#3}{f}{}{}%
      \endgre@style@additionalstafflines%
      \gre@dimen@glyphraisevalue=\gre@dimen@savedglyphraise\relax%
    \fi %
  \fi %
  \relax %
}%

% #1 0 for over staff,
%    1 for under staff
% #2 length of line
% #3 0 for no space before,
%    1 for \gre@space@dimen@additionallineswidth before,
%    2 for #4 space before
% #4 custom space before, used if #2 is 2
% #5 0 for no space after,
%    1 for \gre@space@dimen@additionallineswidth after,
%    2 for #6 space after
% #6 custom space after, used if #4 is 2
\def\GreDrawAdditionalLine#1#2#3#4#5#6{%
  \ifgre@showlines %
    \xdef\gre@dimen@savedglyphraise{\the\gre@dimen@glyphraisevalue}%
    \gre@style@additionalstafflines %
    \gre@drawadditionalline{#1}{#2}{#3}{#4}{#5}{#6}%
    \endgre@style@additionalstafflines%
    \gre@dimen@glyphraisevalue=\gre@dimen@savedglyphraise\relax%
  \fi %
  \relax %
}
\def\gre@drawadditionalline#1#2#3#4#5#6{%
  \ifcase#3 % 0
    \gre@dimen@temp@five=0pt\relax %
  \or % 1
    \gre@dimen@temp@five=\gre@space@dimen@additionallineswidth\relax %
  \or % 2
    \gre@dimen@temp@five=#4\relax %
  \fi %
  \ifcase#5 % 0
    \gre@dimen@temp@four=0pt\relax %
  \or % 1
    \gre@dimen@temp@four=\gre@space@dimen@additionallineswidth\relax %
  \or % 2
    \gre@dimen@temp@four=#6\relax %
  \fi %
  \gre@dimen@temp@two=%
    \dimexpr(#2 %
    + \gre@dimen@temp@five %
    + \gre@dimen@temp@four)\relax %
  \ifcase#1 % 0
    \gre@dimen@glyphraisevalue=%
      \dimexpr(\gre@dimen@additionalbottomspace %
      + \gre@space@dimen@spacebeneathtext %
      + \gre@space@dimen@spacelinestext %
      + \gre@dimen@currenttranslationheight %
      + \gre@stafflines\gre@dimen@interstafflinespace %
      + \gre@stafflines\gre@dimen@stafflineheight)\relax%
  \or % 1
    \gre@dimen@glyphraisevalue=%
      \dimexpr(\gre@dimen@additionalbottomspace %
      + \gre@space@dimen@spacebeneathtext %
      + \gre@space@dimen@spacelinestext %
      + \gre@dimen@currenttranslationheight %
      - \gre@dimen@interstafflinespace %
      - \gre@dimen@stafflineheight)\relax%
  \fi %
  \raise\gre@dimen@glyphraisevalue\hbox to 0pt{%
    \kern-\gre@dimen@temp@five %
    \vrule height \gre@dimen@stafflineheight %
           width \gre@dimen@temp@two %
  }%
}

% a function to typeset a horizontal line (additional line or episema).
% This function must be called after a call to \GreGlyph.
% #1 is the letter of the height of the episema (not the height of the note
% it corresponds to.
% #2 is note position case as in the table above
% #3 is the ambitus for a two note episema at the diagonal stroke of a
%    porrectus, porrectus flexus, orculus resupinus, or torculus resupinus
%    flexus
% #4 is 0 for an horizontal episema, 1 for an horizontal episema under a
%    note, 3 for a line at the bottom, 2 for a line at the top
% #5 is f for a normal episema, l for a small episema aligned left,
%    c for a small episema aligned center, or r for a small episema
%    aligned right
% #6 is the vertical nudge
% #7 is horizontal episema interline position case
%%   0 : auto
%%   1 : middle
%%   2 : low in space above note
%%   3 : high in space above note
%%   4 : low in space below note
%%   5 : high in space below note
\def\gre@hepisorline#1#2#3#4#5#6#7{{%
  \ifcase#4 % 0
    \gre@calculate@glyphraisevalue{#1}{9}{#7}%
  \or % 1
    \gre@calculate@glyphraisevalue{#1}{5}{#7}%
  \or % 2
    % the glyphraisevalue is ignored anyway... but it's just in case...
    \gre@calculate@glyphraisevalue{\gre@pitch@ledger@above}{0}{}%
  \or % 3
    \gre@calculate@glyphraisevalue{\gre@pitch@ledger@below}{0}{}%
  \fi %
  \global\gre@dimen@glyphraisevalue=\dimexpr\gre@dimen@glyphraisevalue#6\relax %
  \gre@h@case{#2}{#3}{#4}{#5}%
  \relax%
}}%

% dumb top function
% #6 is a trick for bridges: if we must use a different height because of a
%    bridge, use #6, otherwise use #1
% #7 is used for setting heuristics
% #8 is the vertical nudge
% #9 is horizontal episema interline position case
%%   0 : auto
%%   1 : middle
%%   2 : low in space above note
%%   3 : high in space above note
%%   4 : low in space below note
%%   5 : high in space below note
\def\GreHEpisema#1#2#3#4#5#6#7#8#9{%
  \ifgre@boxing\else\ifgre@disablehepisema\else %
    \gre@prephepisemaledgerlineheuristics%
    #7%
    \ifgre@hepisemabridge%
      \gre@hepisorline{#6}{#2}{#3}{#4}{#5}{#8}{#9}%
    \else %
      \gre@hepisorline{#1}{#2}{#3}{#4}{#5}{#8}{#9}%
    \fi %
    \gre@resetledgerlineheuristics%
  \fi\fi %
  \relax %
}%

\newif\ifgre@hepisemabridge
\gre@hepisemabridgetrue

\def\gresethepisema#1{%
  \IfStrEqCase{#1}{%
    {bridge}%
      {\gre@hepisemabridgetrue}%
    {break}%
      {\gre@hepisemabridgefalse}%
    }[% all other cases
      \gre@error{Unrecognized option, #1, for \protect\gresethepisema\MessageBreak Possible options are: 'bridge' and 'break'}%
    ]%
}%

% same but for a "bridge episema" after the last note of a glyph (element, syllable) if the next episema is at the same height
% #1 is the height
% #2 is 0 for episema above, 1 for episema below
% #3 is the "space case" for the following note, if a punctum inclinatum
% #4 is for setting heuristics
% #5 is the vertical nudge
% #6 is horizontal episema interline position case
%%   0 : auto
%%   1 : middle
%%   2 : low in space above note
%%   3 : high in space above note
%%   4 : low in space below note
%%   5 : high in space below note
\def\GreHEpisemaBridge#1#2#3#4#5#6{%
  \ifgre@hepisemabridge%
    \ifgre@boxing\else %
      \gre@prephepisemaledgerlineheuristics%
      #4%
      \ifcase#2 %
        \gre@calculate@glyphraisevalue{#1}{9}{#6}%
      \or %
        \gre@calculate@glyphraisevalue{#1}{5}{#6}%
      \fi %
      \global\gre@dimen@glyphraisevalue=\dimexpr\gre@dimen@glyphraisevalue#5\relax %
      \IfStrEq{-1}{#3}{%
        \raise\gre@dimen@glyphraisevalue\hbox to 0pt{\gre@font@music\gre@char@he@punctum{f}\hss}%
      }{%
        % special handling for punctum inclinatum
        \setbox\gre@box@temp@width=\hbox{\gre@font@music\GreCPHEpisemaPunctumReduced}%
        \gre@dimen@temp@four = \wd\gre@box@temp@width\relax %
        \gre@get@spaceskip{#3}%
        % convert rubber length to dimension... not perfect but works in the usual case
        \gre@dimen@temp@three = 1\gre@skip@temp@four\relax %
        \raise\gre@dimen@glyphraisevalue\hbox to 0pt{%
          \loop\ifdim\gre@dimen@temp@three > 0pt\relax%
            \advance\gre@dimen@temp@three by -\gre@dimen@temp@four\relax %
            {\gre@font@music\GreCPHEpisemaPunctumReduced}%
          \repeat %
          \kern\gre@dimen@temp@three{\gre@font@music\GreCPHEpisemaPunctumReduced}%
          \hss %
        }%
      }%
      \gre@resetledgerlineheuristics%
    \fi %
  \fi %
  \relax %
}%

%%%%%%%%%%%%%%%%%%%%%%%%%%%%%%%%%%%%%%
%% macros for the typesetting of bars
%%%%%%%%%%%%%%%%%%%%%%%%%%%%%%%%%%%%%%

% we define two types of macro for each four bar : when it is inside a syllable, and when it is not

\def\GreInVirgula#1#2#3{%
  \gre@writebar{0}{1}{#1}{#2}{#3}%
  \relax%
}%

\def\GreVirgula#1#2#3{%
  \gre@writebar{0}{0}{#1}{#2}{#3}%
  \relax%
}%

\def\GreInDivisioMinima#1#2#3{%
  \gre@writebar{1}{1}{#1}{#2}{#3}%
  \relax%
}%

\def\GreDivisioMinima#1#2#3{%
  \gre@writebar{1}{0}{#1}{#2}{#3}%
  \relax%
}%

\def\GreInDivisioMinor#1#2{%
  \gre@writebar{2}{1}{#1}{#2}{0}%
  \relax%
}%

\def\GreDivisioMinor#1#2{%
  \gre@writebar{2}{0}{#1}{#2}{0}%
  \relax%
}%

\def\GreInDivisioMaior#1#2{%
  \gre@writebar{3}{1}{#1}{#2}{0}%
  \relax%
}%

\def\GreDivisioMaior#1#2{%
  \gre@writebar{3}{0}{#1}{#2}{0}%
  \relax%
}%

\def\GreDominica#1#2#3{%
  \ifcase#1\or %
    \gre@writebar{6}{0}{#2}{#3}{0}%
  \or %
    \gre@writebar{7}{0}{#2}{#3}{0}%
  \or %
    \gre@writebar{8}{0}{#2}{#3}{0}%
  \or %
    \gre@writebar{9}{0}{#2}{#3}{0}%
  \or %
    \gre@writebar{10}{0}{#2}{#3}{0}%
  \or %
    \gre@writebar{11}{0}{#2}{#3}{0}%
  \or %
    \gre@writebar{12}{0}{#2}{#3}{0}%
  \or %
    \gre@writebar{13}{0}{#2}{#3}{0}%
  \fi %
  \relax%
}%

\def\GreInDominica#1#2#3{%
  \ifcase#1\or %
    \gre@writebar{6}{1}{#2}{#3}{0}%
  \or %
    \gre@writebar{7}{1}{#2}{#3}{0}%
  \or %
    \gre@writebar{8}{1}{#2}{#3}{0}%
  \or %
    \gre@writebar{9}{1}{#2}{#3}{0}%
  \or %
    \gre@writebar{10}{1}{#2}{#3}{0}%
  \or %
    \gre@writebar{11}{1}{#2}{#3}{0}%
  \or %
    \gre@writebar{12}{1}{#2}{#3}{0}%
  \or %
    \gre@writebar{13}{1}{#2}{#3}{0}%
  \fi %
  \relax%
}%

\def\GreInDivisioFinalis#1#2{%
  \ifgre@endofscore %
    \gre@writebar{5}{1}{#1}{#2}{0}%
  \else %
    \gre@writebar{4}{1}{#1}{#2}{0}%
  \fi %
  \relax%
}%

\def\GreDivisioFinalis#1#2{%
  \ifgre@endofscore %
    \gre@writebar{5}{0}{#1}{#2}{0}%
  \else %
    \gre@writebar{4}{0}{#1}{#2}{0}%
  \fi %
  \relax%
}%

% #1 is #2 of gre@writebar (@standalone or nothing)
% #2 is #3 of gre@writebar (@text or @notext, only if standalone)
% #3 is #5 of gre@writebar (@short or nothing)
% emits the suffix of the space
\def\gre@bar@space@suffix#1#2#3{%
  \ifgre@newbarspacing %
    \ifcase#1 %
      \ifcase#3 %
        \ifcase#2 @standalone@notext\or @standalone@text\fi %
      \or %
        \ifcase#2 @standalone@notext@short\or @standalone@text@short\fi %
      \fi %
    \else %
      \ifcase#3 \or @short\fi %
    \fi %
  \else %
    \ifcase#3 \or @short\fi %
  \fi %
}%

\newcount\gre@count@shiftaftermora
\gre@count@shiftaftermora=5

\def\gresetshiftaftermora#1{%
  \IfStrEqCase{#1}{%
    {always}%
      {\gre@count@shiftaftermora=5}%
    {notesonly}%
      {\gre@count@shiftaftermora=4}%
    {barsonly}%
      {\gre@count@shiftaftermora=3}%
    {barsnotextonly}%
      {\gre@count@shiftaftermora=2}%
    {barsinsideonly}%
      {\gre@count@shiftaftermora=1}%
    {never}%
      {\gre@count@shiftaftermora=0}%
    }[% all other cases
      \gre@error{Unrecognized option "#1" in gresetshiftaftermora\MessageBreak Possible options are: 'always', 'barsonly', 'notesonly', 'barsnotextonly', 'barsinsideonly' and 'never'}%
    ]%
}%

% width of a punctum mora, reinitalized at each score
\newdimen\gre@dimen@morawidth

% sets gre@skip@punctummorashift to the (usually negative) shift to apply before
% a syllable following a punctum mora, to cancel its taking into account into
% horizontal placement.
% First argument is:
%   - 1 for the general case
%   - 2 when the punctum mora is before a bar
\def\gre@get@unkern@aftermora#1{%
  \ifdim\gre@dimen@morawidth=0pt\relax %
    \setbox\gre@box@temp@width=\hbox{\gre@font@music\gre@fontchar@punctummora}%
    \global\gre@dimen@morawidth=\wd\gre@box@temp@width %
  \fi %
  \ifnum #1=1%
    \global\gre@skip@punctummorashift=\glueexpr - \gre@dimen@morawidth - \gre@space@skip@spacebeforesigns + \gre@space@skip@moraadjustment\relax %
  \else %
    \global\gre@skip@punctummorashift=\glueexpr - \gre@dimen@morawidth - \gre@space@skip@spacebeforesigns + \gre@space@skip@moraadjustmentbar\relax %
  \fi %
  \relax %
}

\def\gre@unkern@bar@aftermora{%
  \ifgre@newbarspacing%
    % don't apply a negative kern on the first glyph
    % while boxing, otherwise the box width will be wrong
    \ifgre@firstglyph %
      \ifgre@boxing\else %
        \gre@get@unkern@aftermora{2}%
        \kern\gre@skip@punctummorashift %
      \fi %
    \else %
      \gre@get@unkern@aftermora{2}%
      \kern\gre@skip@punctummorashift %
    \fi %
  \else%
    \gre@get@unkern@aftermora{2}%
    \kern\gre@skip@punctummorashift %
  \fi%
  \relax %
}

\newskip\gre@skip@bar@lastskip% skip after last bar

%a macro to write a bar
%% 1: the type of the bar : 0 for virgula, 1 for minima 2 for minor, 3 for major, 4 for finalis and 5 for the last finalis, 6 to 13 for dominican bars
%% 2: is 0 if it is in a syllable containing only this bar, 1 otherwise
%% 3: is 0 if there's no text under the bar or 1 if there is text under the bar
%% 4: macros that may happen before the skip after the bar (typically GreVEpisema)
%% 5: 0 for the normal version of spaces, 1 for the short one (virgula and minima only)
\def\gre@writebar#1#2#3#4#5{%
  % first, for the bar to be really centered, if the last glyph has a punctum
  % mora, we kern of the corresponding space. We do it only in the case
  % of a bar in the middle of other notes.
  \ifgre@lastendswithmora %
    \ifnum\gre@count@shiftaftermora=0\else\ifnum\gre@count@shiftaftermora=4\else %
      % if bar doesn’t has its own syllable, then we need to take care of the kern after a punctum mora here
      \ifcase #2\or %
        \gre@unkern@bar@aftermora %
      \fi %
    \fi\fi %
  \fi %
  \gre@newglyphcommon %
  \gre@calculate@glyphraisevalue{\gre@pitch@bar}{0}{}% bar glyphs are made to be at this height
  \GreNoBreak %
  % count indicating if we have to output space or not:
  \gre@count@temp@one=0\relax %
  \ifnum#2=1\relax %
    \gre@count@temp@one=1\relax %
  \else %
    \ifgre@newbarspacing %
      \gre@count@temp@one=1\relax %
    \fi %
  \fi %
  \gre@skip@temp@four=0pt\relax %
  \ifcase#1 % 0 : virgula
    \ifnum\gre@count@temp@one=1\relax %
      \gre@skip@temp@four = \csname gre@space@skip@bar@virgula\gre@bar@space@suffix{#2}{#3}{#5}\endcsname\relax%
      \gre@hskip\gre@skip@temp@four %
      \GreNoBreak %
    \fi %
    \setbox\gre@box@temp@width=\hbox{\gre@font@music\GreCPVirgula}%
    \raise\gre@dimen@glyphraisevalue\hbox{\gre@font@music\GreCPVirgula}%
    #4\relax %
    \ifnum\gre@count@temp@one=1\relax %
      \GreNoBreak %
      \gre@skip@temp@four = \csname gre@space@skip@bar@virgula\gre@bar@space@suffix{#2}{#3}{#5}\endcsname\relax%
      \gre@hskip\gre@skip@temp@four %
    \fi %
  \or % 1 : minima
    \ifnum\gre@count@temp@one=1\relax %
      \gre@skip@temp@four = \csname gre@space@skip@bar@minima\gre@bar@space@suffix{#2}{#3}{#5}\endcsname\relax%
      \gre@hskip\gre@skip@temp@four %
      \GreNoBreak %
    \fi %
    \setbox\gre@box@temp@width=\hbox{\gre@font@music\GreCPDivisioMinima}%
    \raise\gre@dimen@glyphraisevalue\hbox{\gre@font@music\GreCPDivisioMinima}%
    #4\relax %
    \ifnum\gre@count@temp@one=1\relax %
      \GreNoBreak %
      \gre@skip@temp@four = \csname gre@space@skip@bar@minima\gre@bar@space@suffix{#2}{#3}{#5}\endcsname\relax%
      \gre@hskip\gre@skip@temp@four %
    \fi %
  \or % 2 : minor
    \ifnum\gre@count@temp@one=1\relax %
      \gre@skip@temp@four = \csname gre@space@skip@bar@minor\gre@bar@space@suffix{#2}{#3}{#5}\endcsname\relax%
      \gre@hskip\gre@skip@temp@four %
      \GreNoBreak %
    \fi %
    \setbox\gre@box@temp@width=\hbox{\gre@font@music\GreCPDivisioMinor}%
    \raise\gre@dimen@glyphraisevalue\hbox{\gre@font@music\GreCPDivisioMinor}%
    #4\relax %
    \ifnum\gre@count@temp@one=1\relax %
      \GreNoBreak %
      \gre@skip@temp@four = \csname gre@space@skip@bar@minor\gre@bar@space@suffix{#2}{#3}{#5}\endcsname\relax%
      \gre@hskip\gre@skip@temp@four %
    \fi %
  \or % 3 : maior
    \ifnum\gre@count@temp@one=1\relax %
      \gre@skip@temp@four = \csname gre@space@skip@bar@maior\gre@bar@space@suffix{#2}{#3}{#5}\endcsname\relax%
      \gre@hskip\gre@skip@temp@four %
      \GreNoBreak %
    \fi %
    \setbox\gre@box@temp@width=\hbox{\gre@font@music\GreCPDivisioMaior}%
    \gre@fontchar@divisiomaior %
    #4\relax %
    \ifnum\gre@count@temp@one=1\relax %
      \GreNoBreak %
      \gre@skip@temp@four = \csname gre@space@skip@bar@maior\gre@bar@space@suffix{#2}{#3}{#5}\endcsname\relax%
      \gre@hskip\gre@skip@temp@four %
    \fi %
  \or % 4 : finalis
    \ifnum\gre@count@temp@one=1\relax %
      \gre@skip@temp@four = \csname gre@space@skip@bar@finalis\gre@bar@space@suffix{#2}{#3}{#5}\endcsname\relax%
      \gre@hskip\gre@skip@temp@four %
      \GreNoBreak %
    \fi %
    \setbox\gre@box@temp@width=\hbox{\gre@fontchar@divisiofinalis}%
    #4\relax %
    \gre@fontchar@divisiofinalis%
    \ifnum\gre@count@temp@one=1\relax %
      \GreNoBreak %
      \gre@skip@temp@four = \csname gre@space@skip@bar@finalis\gre@bar@space@suffix{#2}{#3}{#5}\endcsname\relax%
      \gre@hskip\gre@skip@temp@four %
    \fi %
  \or % 5 : final finalis
    \ifnum\gre@count@temp@one=1\relax %
      \gre@skip@temp@four = \csname gre@space@skip@bar@finalfinalis\gre@bar@space@suffix{#2}{#3}{#5}\endcsname\relax%
      \gre@hskip\gre@skip@temp@four %
      \GreNoBreak %
    \fi %
    \setbox\gre@box@temp@width=\hbox{\gre@fontchar@divisiofinalis}%
    #4\relax %
    \gre@fontchar@divisiofinalis%
    \ifgre@newbarspacing %
      \GreNoBreak %
      \gre@skip@temp@four = \csname gre@space@skip@bar@finalfinalis\gre@bar@space@suffix{#2}{#3}{#5}\endcsname\relax%
      \gre@hskip\gre@skip@temp@four %
      \GreNoBreak %
    \fi %
  \or % 6 : dominican bar 1
    \gre@calculate@glyphraisevalue{\gre@pitch@e}{0}{}%
    % we need to adjust the height of the bar a little so that it is perfectly aligned with the bottom (or the top for some bars) of the staff line, which is not the case by default if \gre@stafflinefactor is not 10.
    \advance\gre@dimen@glyphraisevalue by -\gre@dimen@stafflinediff\relax%
    \ifnum\gre@count@temp@one=1\relax %
      \gre@skip@temp@four = \csname gre@space@skip@bar@dominican\gre@bar@space@suffix{#2}{#3}{#5}\endcsname\relax%
      \gre@hskip\gre@skip@temp@four %
      \GreNoBreak %
    \fi %
    \setbox\gre@box@temp@width=\hbox{\gre@font@music\GreCPDivisioDominican}%
    \raise\gre@dimen@glyphraisevalue\hbox{\gre@font@music\GreCPDivisioDominican}%
    #4\relax %
    \ifnum\gre@count@temp@one=1\relax %
      \GreNoBreak %
      \gre@skip@temp@four = \csname gre@space@skip@bar@dominican\gre@bar@space@suffix{#2}{#3}{#5}\endcsname\relax%
      \gre@hskip\gre@skip@temp@four %
    \fi %
  \or % 7 : dominican bar 2
    \gre@calculate@glyphraisevalue{\gre@pitch@e}{0}{}%
    \advance\gre@dimen@glyphraisevalue by \gre@dimen@stafflinediff\relax%
    \ifnum\gre@count@temp@one=1\relax %
      \gre@skip@temp@four = \csname gre@space@skip@bar@dominican\gre@bar@space@suffix{#2}{#3}{#5}\endcsname\relax%
      \gre@hskip\gre@skip@temp@four %
      \GreNoBreak %
    \fi %
    \setbox\gre@box@temp@width=\hbox{\gre@font@music\GreCPDivisioDominicanAlt}%
    \raise\gre@dimen@glyphraisevalue\hbox{\gre@font@music\GreCPDivisioDominicanAlt}%
    #4\relax %
    \ifnum\gre@count@temp@one=1\relax %
      \GreNoBreak %
      \gre@skip@temp@four = \csname gre@space@skip@bar@dominican\gre@bar@space@suffix{#2}{#3}{#5}\endcsname\relax%
      \gre@hskip\gre@skip@temp@four %
    \fi %
  \or % 8 : dominican bar 3
    \ifnum\gre@count@temp@one=1\relax %
      \gre@skip@temp@four = \csname gre@space@skip@bar@dominican\gre@bar@space@suffix{#2}{#3}{#5}\endcsname\relax%
      \gre@hskip\gre@skip@temp@four %
      \GreNoBreak %
    \fi %
    \advance\gre@dimen@glyphraisevalue by -\gre@dimen@stafflinediff\relax%
    \setbox\gre@box@temp@width=\hbox{\gre@font@music\GreCPDivisioDominican}%
    \raise\gre@dimen@glyphraisevalue\hbox{\gre@font@music\GreCPDivisioDominican}%
    #4\relax %
    \ifnum\gre@count@temp@one=1\relax %
      \GreNoBreak %
      \gre@skip@temp@four = \csname gre@space@skip@bar@dominican\gre@bar@space@suffix{#2}{#3}{#5}\endcsname\relax%
      \gre@hskip\gre@skip@temp@four %
    \fi %
  \or % 9 : dominican bar 4
    \ifnum\gre@count@temp@one=1\relax %
      \gre@skip@temp@four = \csname gre@space@skip@bar@dominican\gre@bar@space@suffix{#2}{#3}{#5}\endcsname\relax%
      \gre@hskip\gre@skip@temp@four %
      \GreNoBreak %
    \fi %
    \advance\gre@dimen@glyphraisevalue by \gre@dimen@stafflinediff\relax%
    \setbox\gre@box@temp@width=\hbox{\gre@font@music\GreCPDivisioDominicanAlt}%
    \raise\gre@dimen@glyphraisevalue\hbox{\gre@font@music\GreCPDivisioDominicanAlt}%
    #4\relax %
    \ifnum\gre@count@temp@one=1\relax %
      \GreNoBreak %
      \gre@skip@temp@four = \csname gre@space@skip@bar@dominican\gre@bar@space@suffix{#2}{#3}{#5}\endcsname\relax%
      \gre@hskip\gre@skip@temp@four %
    \fi %
  \or % 10 : dominican bar 5
    \gre@calculate@glyphraisevalue{\gre@pitch@i}{0}{}%
    \advance\gre@dimen@glyphraisevalue by -\gre@dimen@stafflinediff\relax%
    \ifnum\gre@count@temp@one=1\relax %
      \gre@skip@temp@four = \csname gre@space@skip@bar@dominican\gre@bar@space@suffix{#2}{#3}{#5}\endcsname\relax%
      \gre@hskip\gre@skip@temp@four %
      \GreNoBreak %
    \fi %
    \setbox\gre@box@temp@width=\hbox{\gre@font@music\GreCPDivisioDominican}%
    \raise\gre@dimen@glyphraisevalue\hbox{\gre@font@music\GreCPDivisioDominican}%
    #4\relax %
    \ifnum\gre@count@temp@one=1\relax %
      \GreNoBreak %
      \gre@skip@temp@four = \csname gre@space@skip@bar@dominican\gre@bar@space@suffix{#2}{#3}{#5}\endcsname\relax%
      \gre@hskip\gre@skip@temp@four %
    \fi %
  \or % 11 : dominican bar 6
    \gre@calculate@glyphraisevalue{\gre@pitch@i}{0}{}%
    \advance\gre@dimen@glyphraisevalue by \gre@dimen@stafflinediff\relax%
    \ifnum\gre@count@temp@one=1\relax %
      \gre@skip@temp@four = \csname gre@space@skip@bar@dominican\gre@bar@space@suffix{#2}{#3}{#5}\endcsname\relax%
      \gre@hskip\gre@skip@temp@four %
      \GreNoBreak %
    \fi %
    \setbox\gre@box@temp@width=\hbox{\gre@font@music\GreCPDivisioDominicanAlt}%
    \raise\gre@dimen@glyphraisevalue\hbox{\gre@font@music\GreCPDivisioDominicanAlt}%
    #4\relax %
    \ifnum\gre@count@temp@one=1\relax %
      \GreNoBreak %
      \gre@skip@temp@four = \csname gre@space@skip@bar@dominican\gre@bar@space@suffix{#2}{#3}{#5}\endcsname\relax%
      \gre@hskip\gre@skip@temp@four %
    \fi %
  \or % 12 : dominican bar 7
    \gre@calculate@glyphraisevalue{\gre@pitch@k}{0}{}%
    \advance\gre@dimen@glyphraisevalue by -\gre@dimen@stafflinediff\relax%
    \ifnum\gre@count@temp@one=1\relax %
      \gre@skip@temp@four = \csname gre@space@skip@bar@dominican\gre@bar@space@suffix{#2}{#3}{#5}\endcsname\relax%
      \gre@hskip\gre@skip@temp@four %
      \GreNoBreak %
    \fi %
    \setbox\gre@box@temp@width=\hbox{\gre@font@music\GreCPDivisioDominican}%
    \raise\gre@dimen@glyphraisevalue\hbox{\gre@font@music\GreCPDivisioDominican}%
    #4\relax %
    \ifnum\gre@count@temp@one=1\relax %
      \GreNoBreak %
      \gre@skip@temp@four = \csname gre@space@skip@bar@dominican\gre@bar@space@suffix{#2}{#3}{#5}\endcsname\relax%
      \gre@hskip\gre@skip@temp@four %
    \fi %
  \or % 13 : dominican bar 8
    \gre@calculate@glyphraisevalue{\gre@pitch@k}{0}{}%
    \advance\gre@dimen@glyphraisevalue by \gre@dimen@stafflinediff\relax%
    \ifnum\gre@count@temp@one=1\relax %
      \gre@skip@temp@four = \csname gre@space@skip@bar@dominican\gre@bar@space@suffix{#2}{#3}{#5}\endcsname\relax%
      \gre@hskip\gre@skip@temp@four %
      \GreNoBreak %
    \fi %
    \setbox\gre@box@temp@width=\hbox{\gre@font@music\GreCPDivisioDominicanAlt}%
    \raise\gre@dimen@glyphraisevalue\hbox{\gre@font@music\GreCPDivisioDominicanAlt}%
    #4\relax %
    \ifnum\gre@count@temp@one=1\relax %
      \GreNoBreak %
      \gre@skip@temp@four = \csname gre@space@skip@bar@dominican\gre@bar@space@suffix{#2}{#3}{#5}\endcsname\relax%
      \gre@hskip\gre@skip@temp@four %
    \fi %
  \fi %
  \global\gre@skip@bar@lastskip=\gre@skip@temp@four %
  \gre@debugmsg{spacing}{Width of bar just printed: \the\wd\gre@box@temp@width}%
  \gre@debugmsg{spacing}{Last bar space: \the\gre@skip@temp@four}%
  \global\gre@dimen@lastglyphwidth=\wd\gre@box@temp@width %
  \directlua{gregoriotex.adjust_line_height(\gre@insidediscretionary)}%
  \global\gre@firstglyphfalse %
  \relax%
}%

\def\gre@fontchar@divisiomaior{%
  \ifnum\gre@stafflinefactor=17\relax %
    %\gre@calculate@glyphraisevalue{\gre@pitch@bar}{0}{}% bar glyphs are made to be at this height
    \raise\gre@dimen@glyphraisevalue\hbox{\gre@font@music\GreCPDivisioMaior}%
  \else %
    \setbox\gre@box@temp@width=\hbox{\gre@font@music\GreCPDivisioMaior}%
    % we calculate the raise of the bar
    \gre@dimen@temp@five=\dimexpr(\gre@dimen@additionalbottomspace %
      + \gre@space@dimen@spacebeneathtext %
      + \gre@space@dimen@spacelinestext %
      + \gre@dimen@currenttranslationheight)\relax%
    % we calculate the height of the bar
    \raise\gre@dimen@temp@five\hbox{\vrule height \gre@dimen@staffheight width \wd\gre@box@temp@width}%
  \fi %
  \relax %
}%

\def\gre@fontchar@divisiofinalis{%
  \gre@calculate@glyphraisevalue{\gre@pitch@bar}{0}{}% bar glyphs are made to be at this height
  \gre@fontchar@divisiomaior %
  \kern\gre@space@dimen@divisiofinalissep %
  \GreNoBreak %
  \gre@fontchar@divisiomaior %
}%

% Flag to tell if we have to keep the localrightbox until the end
\newif\ifgre@keeprightbox%

%macro to end a line with a divisio finalis
\def\GreFinalDivisioFinalis#1{%
  \GreBarSyllable{\GreSetThisSyllable{}{}{}{}{}}{}{}{1}{\GreSetNextSyllable{}{}{}{}{}\GreLastOfLine}{}{16}{}{%
    \ifgre@newbarspacing\else %
      \gre@hskip\gre@space@skip@bar@finalfinalis %
      \GreNoBreak %
    \fi %
    \GreLastOfScore %
    \GreDivisioFinalis{0}{}%
    #1%
  }%
  \relax%
}%

%macro to end a line with a divisio maior
\def\GreFinalDivisioMaior#1{%
  \GreBarSyllable{\GreSetThisSyllable{}{}{}{}{}}{}{}{1}{\GreSetNextSyllable{}{}{}{}{}\GreLastOfLine}{}{16}{}{%
    \GreLastOfScore %
    \GreDivisioMaior{0}{}%
    #1%
  }%
  \relax%
}%

%%%%%%%%%%%%%%%%%%%%%%%%%%%%%%%%%%%%%%%%%%%
%% macros for filling holes of empty notes
%%%%%%%%%%%%%%%%%%%%%%%%%%%%%%%%%%%%%%%%%%%

% flag to indicate that lines behind a punctum cavum should be hidden
\newif\ifgre@hidepclines
% default state is to not hide them
\gre@hidepclinesfalse

% macro for manipulating above flag
\def\gresetlinesbehindpunctumcavum#1{%
  \IfStrEqCase{#1}{%
    {visible}%
      {\gre@hidepclinesfalse}%
    {invisible}%
      {\gre@hidepclinestrue}%
    }[% all other cases
      \gre@error{Unrecognized option "#1" for \protect\gresetlinesbehindpunctumcavum\MessageBreak Possible options are: 'visible' and 'invisible'}%
    ]%
}%

% flag to indicate that lines behind an alteration should be hidden
\newif\ifgre@hidealtlines
% default state is to not hide them
\gre@hidealtlinesfalse

% macro for manipulating above flag
\def\gresetlinesbehindalteration#1{%
  \IfStrEqCase{#1}{%
    {visible}%
      {\gre@hidealtlinesfalse}%
    {invisible}%
      {\gre@hidealtlinestrue}%
    }[% all other cases
      \gre@error{Unrecognized option "#1" for \protect\gresetlinesbehindalteration\MessageBreak Possible options are: 'visible' and 'invisible'}%
    ]%
}%

% is last glyph a cavum?
\newcount\gre@count@lastglyphiscavum
% 0 if no
% 1 if current glyph is a cavum (set once the cavum is filled)
% 2 if previous glyph is a cavum (which is the most interesting information)
\gre@count@lastglyphiscavum=0

% #1 is the character with which we fill the hole, and we suppose that
% isonaline and glyphraisevalue are correctly set.
% #2 is 0 for alteration or 1 for normal cavum
\def\gre@fillhole#1#2{%
  \ifgre@boxing\else %
    \global\gre@count@lastglyphiscavum=1\relax %
    \setbox\gre@box@temp@sign=\hbox{#1}%
    \ifcase#2\raise \gre@dimen@glyphraisevalue\fi%
    \hbox to 0pt{%
      {%
      \color{grebackgroundcolor}%
      \copy\gre@box@temp@sign %
      }%
      %\pdfliteral{}% this is a ugly hack for old versions of LuaTeX to work
      \hss %
    }%
    \GreNoBreak %
  \fi %
  \relax %
}%

%%%%%%%%%%%%%%%%%%%%%%%%%%%%%%%%%%%%%%
%% macros for typesetting alterations
%%%%%%%%%%%%%%%%%%%%%%%%%%%%%%%%%%%%%%

% the top level macro:
% #1 is the height
% #2 is the char of the alteration
% #3 is the char of the alteration hole
% #4 is 1 if the alteration is the flat in a clef
% #5 are the signs to typeset before the glyph (typically additional bars, as they must be "behind" the glyph)
% #6 are the signs to typeset after the glyph (almost all signs)
% #7 is the line:char:column for a textedit link
\def\gre@alteration#1#2#3#4#5#6#7{%
  \setbox\gre@box@temp@width=\hbox{\gre@pointandclick{#2}{#7}}%
  \ifnum#4=0\relax %
    \gre@newglyphcommon %
  \fi %
  \global\gre@dimen@lastglyphwidth=\wd\gre@box@temp@width %
  % the three next lines are a trick to get the additional lines below the glyphs
  \gre@skip@temp@one = \gre@dimen@lastglyphwidth\relax%
  \kern\gre@skip@temp@one %
  #5\relax %
  \kern-\gre@skip@temp@one %
  \gre@calculate@glyphraisevalue{#1}{0}{}%
  \ifgre@hidealtlines %
    \gre@fillhole{#3}{0}%
  \fi %
  \raise \gre@dimen@glyphraisevalue%
  \copy\gre@box@temp@width%
  #6\relax %
  \ifnum#4=0\relax %
    % we try to avoid line breaking after a flat or a natural
    \GreNoBreak %
    \gre@skip@temp@four = \gre@space@dimen@alterationspace\relax%
    \ifgre@firstglyph%
      \gre@debugmsg{bolshift}{making adjustments for leading alteration}%
      \global\advance\gre@dimen@notesaligncenter by %
        \dimexpr(\wd\gre@box@temp@width + \dimexpr\gre@skip@temp@four)\relax %
      \kern\gre@skip@temp@four %
      \global\gre@dimen@bolextra = \dimexpr(\wd\gre@box@temp@width + \dimexpr\gre@skip@temp@four)\relax%
      \gre@debugmsg{bolshift}{bolextra: \the\gre@dimen@bolextra}%
    \else %
      \gre@hskip\gre@skip@temp@four %
    \fi %
    \GreNoBreak %
    \directlua{gregoriotex.adjust_line_height(\gre@insidediscretionary)}%
  \fi %
  \relax %
}%

% This macro typesets a flat on the height provided by #1.
% #2 is 1 if the flat is part of a clef.

\def\GreFlat#1#2#3#4#5{%
  \gre@alteration{#1}{\gre@fontchar@flat}{\gre@fontchar@flathole}{#2}{#3}{#4}{#5}%
  \relax%
}%

% Same as the one before, but for naturals.

\def\GreNatural#1#2#3#4#5{%
  \gre@alteration{#1}{\gre@fontchar@natural}{\gre@fontchar@naturalhole}{#2}{#3}{#4}{#5}%
  \relax%
}%

% Same as the one before, but for sharps.

\def\GreSharp#1#2#3#4#5{%
  \gre@alteration{#1}{\gre@fontchar@sharp}{\gre@fontchar@sharphole}{#2}{#3}{#4}{#5}%
  \relax%
}%

%%%%%%%%%%%%%%%%%%%%%%%%%%%%%%%%%%%%%%%%
%% macros for typesetting punctum cavum
%%%%%%%%%%%%%%%%%%%%%%%%%%%%%%%%%%%%%%%%

\def\gresetpunctumcavum#1{%
  \IfStrEqCase{#1}{%
    {alternate}%
      {%
        \grechangecavumglyph{Punctum}{greciliae-hollow}{.caeciliae}[greciliae-hole][.caeciliae]%
        \grechangecavumglyph{LineaPunctum}{greciliae-hollow}{.caeciliae}[greciliae-hole][.caeciliae]%
      }%
    {normal}%
      {%
        \greresetcavumglyph{Punctum}%
        \greresetcavumglyph{LineaPunctum}%
      }%
    }[% all other cases
      \gre@error{Unrecognized option "#1" for \protect\gresetpunctumcavum\MessageBreak Possible options are: 'alternate' and 'normal'}%
    ]%
}%

\def\GreCavum#1{%
  \ifgre@hidepclines%
    \gre@fillhole{{\gre@font@music@hole\csname GreHoleCP#1\endcsname}}{1}%
  \fi %
  {\gre@font@music@hollow\csname GreHollowCP#1\endcsname}%
}%

% #1 is 0 for left or 1 for right bracket
% #2 is the height of the lowest note within the brackets
% #3 is the height of the highest note within the brackets
% #4 is the point-and-click string
\def\GreBracket#1#2#3#4{%
  \gre@newglyphcommon %
  \ifcase\numexpr#1\relax\else\GreNoBreak\fi% no break before right bracket
  \setbox\gre@box@temp@width=\hbox{\gre@pointandclick{\gre@font@music %
    \csname GreCPBracket%
      \ifcase\numexpr#1\relax Left\else Right\fi%
      % Logic:
      % if low pitch is on a line or below the staff:
      %   if high pitch is on a line or above the staff:
      %     use short bracket
      %   else:
      %     use normal bracket
      % else:
      %   if high pitch is on a line or above the staff:
      %     use normal bracket
      %   else:
      %     use long bracket
      \ifnum\numexpr#2\relax>\numexpr\gre@pitch@belowstaff\relax % low pitch not below staff
        % numexpr+/ rounds rather than truncates
        \ifnum\numexpr#2/2\relax=\numexpr(#2+1)/2\relax % =odd, low pitch not on a line
          \ifnum\numexpr#3\relax<\numexpr\gre@pitch@abovestaff\relax % high pitch not above staff
            \ifnum\numexpr#3/2\relax=\numexpr(#3+1)/2\relax % =odd, high pitch not on a line
              Long%
            \fi %
          \fi %
        \else % =even, low pitch on a line
          \ifnum\numexpr#3\relax<\numexpr\gre@pitch@abovestaff\relax % high pitch not above staff
            \ifnum\numexpr#3/2\relax=\numexpr(#3+1)/2\relax\else % =even, high pitch on a line
              Short%
            \fi %
          \else % high pitch above staff
            Short%
          \fi %
        \fi %
      \else % low pitch below staff
        % same as above!
        \ifnum\numexpr#3\relax<\numexpr\gre@pitch@abovestaff\relax % high pitch not above staff
          \ifnum\numexpr#3/2\relax=\numexpr(#3+1)/2\relax\else % =even, high pitch on a line
            Short%
          \fi %
        \else % high pitch above staff
          Short%
        \fi %
      \fi %
      \ifcase\numexpr#3-#2\relax %
          Zero%
      \or One%
      \or Two%
      \or Three%
      \or Four%
      \or Five%
      \or Six%
      \or Seven%
      \or Eight%
      \or Nine%
      \or Ten%
      \or Eleven%
      \or Twelve%
      \or Thirteen%
      \or Fourteen%
      \fi%
    \endcsname %
  }{#4}}%
  \global\gre@dimen@lastglyphwidth=\wd\gre@box@temp@width %
  \gre@calculate@glyphraisevalue{#2}{17}{}%
  \raise\gre@dimen@glyphraisevalue%
  \copy\gre@box@temp@width%
  \ifcase\numexpr#1\relax
    \GreNoBreak% no break after left bracket
    \ifgre@firstglyph%
      \gre@debugmsg{bolshift}{making adjustments for leading bracket}%
      \global\advance\gre@dimen@notesaligncenter by \wd\gre@box@temp@width\relax %
      \global\gre@dimen@bolextra = \wd\gre@box@temp@width\relax%
      \gre@debugmsg{bolshift}{bolextra: \the\gre@dimen@bolextra}%
    \fi %
    \GreNoBreak% no break after left bracket
  \fi %
  \directlua{gregoriotex.adjust_line_height(\gre@insidediscretionary)}%
  \gre@endofglyphcommon %
  \relax%
}%
