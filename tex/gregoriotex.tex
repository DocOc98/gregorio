% GregorioTeX boostrap file for Plain TeX
%
% Copyright (C) 2015 The Gregorio Project (see CONTRIBUTORS.md)
%
% This file is part of Gregorio.
%
% Gregorio is free software: you can redistribute it and/or modify
% it under the terms of the GNU General Public License as published by
% the Free Software Foundation, either version 3 of the License, or
% (at your option) any later version.
%
% Gregorio is distributed in the hope that it will be useful,
% but WITHOUT ANY WARRANTY; without even the implied warranty of
% MERCHANTABILITY or FITNESS FOR A PARTICULAR PURPOSE.  See the
% GNU General Public License for more details.
%
% You should have received a copy of the GNU General Public License
% along with Gregorio.  If not, see <http://www.gnu.org/licenses/>.

% this file contains definitions for lines, initial, fonts, etc.


% This file needs to be marked with the version number.  For now I've done this with the following comment, but we should check to see if PlainTeX has something similar to the version declaration of LaTeX and use that if it does.
% 		[2015/08/26 v4.0.0-beta2 GregorioTeX system.]% PARSE_VERSION_DATE_LTX


\edef\greoldcatcode{\the\catcode`@}
\catcode`\@=11

\input ifluatex.sty%
\input luatexbase.sty%
\input luamplib.sty%
\input luaotfload.sty%
\input graphicx.tex % for \resizebox
\input xstring.sty%

\def\gre@error#1{\begingroup%
\def\MessageBreak{^^J}%
\let\protect\string%
\errmessage{GregorioTeX error: #1}%
\endgroup}%

\def\gre@warning#1{\begingroup%
\def\MessageBreak{^^J}%
\let\protect\string%
\message{GregorioTeX warning: #1}%
\endgroup}%

\def\gre@bug#1{\begingroup%
\def\MessageBreak{^^J}%
\let\protect\string%
\errmessage{GregorioTeX bug: #1 !! This is a bug in Gregorio.  Please report it at https://github.com/gregorio-project/gregorio/issues}%
\endgroup}%

\def\gre@typeout#1{\begingroup%
\def\MessageBreak{^^J}%
\let\protect\string%
\message{#1}%
\endgroup}%

\ifcsname gre@debug\endcsname%
\else%
  \def\gre@debug{}%
\fi%

\ifcsname f@size\endcsname%
\else%
  \def\f@size{\directlua{gregoriotex.font_size()}}%
\fi%

\newif\ifgre@allowdeprecated%
\gre@allowdeprecatedtrue%

\long\def\gre@metapost#1{{%
  \gre@localleftbox{}%
  \gre@localrightbox{}%
  \mplibcode
  #1
  \endmplibcode%
}}%

\long\def\gre@iflatex#1{}%
\input gregoriotex-main.tex

%%%%%%%%%
%% PlainTeX specific definitions for fonts
%%%%%%%%%

\def\GreBold#1{%
  {\bf #1}%
  \relax %
}%

\def\GreItalic#1{%
  {\it #1}%
  %\relax 
}%

\def\GreSmallCaps#1{%
  {\sc #1}%
  \relax %
}%

\def\GreTypewriter#1{%
  {\tt #1}%
  \relax %
}%

% Note since underlining and coloring don’t exist in PlainTeX, but we need the following commands to maintain compatibility with LaTeX.
\def\GreUnderline#1{%
  \gre@warning{Underlining cannot be done in PlainTeX}%
  {#1}%
  \relax %
}%

\def\GreColored#1{%
  \gre@warning{Text color cannot be changed in PlainTeX}%
  {#1}%
  \relax %
}%

%%%%%%%%%%%%%%%
%% Line color
%%%%%%%%%%%%%%%

% Colors don't work in PlainTeX, but we need these functions for interoperability with LaTeX.

\def\gresetlinecolor#1{%
  \gre@warning{Line color cannot be changed in PlainTeX}%
  \relax%
}


\def\grecoloredlines#1#2#3{%
  \gre@deprecated{\protect\grecoloredlines}{\protect\gresetlinecolor}%
  \gresetlinecolor{}%
  \relax %
}

\def\greredlines{%
  \gre@deprecated{\protect\greredlines}{\protect\gresetlinecolor{gregoriocolor}}%
  \gresetlinecolor{}%
  \relax %
}

\let\redlines\greredlines

\def\grenormallines{%
  \gre@deprecated{\protect\grenormallines}{\protect\gresetlinecolor{black}}%
  \gresetlinecolor{}%
  \relax %
}

\let\normallines\grenormallines

%%%%%%%%%%%%%%%%%%%%
%% Formatting environments
%%%%%%%%%%%%%%%%%%%%

\font\gre@font@biginitial=pncr at 80pt\relax%
\font\gre@font@initial=pncr at 40pt\relax%

\def\gre@style@initial{%
  \begingroup%
  \gre@font@initial%
  \relax %
}%
\def\endgre@style@initial{\endgroup}%

\def\gre@style@biginitial{%
  \begingroup%
  \gre@font@biginitial%
  \relax %
}%
\def\endgre@style@biginitial{\endgroup}%

\def\gre@style@translation{%
  \begingroup%
  \it%
}%
\def\endgre@style@translation{\endgroup}%

\def\gre@style@abovelinestext{%
  \begingroup%
  \it%
}%
\def\endgre@style@abovelinestext{\endgroup}%

\def\gre@style@normalstafflines{%
  \begingroup%
  \relax%
}%
\def\endgre@style@normalstafflines{\endgroup}%

\def\gre@style@additionalstafflines{%
  \begingroup%
  \gre@style@normalstafflines%
}%
\def\endgre@style@additionalstafflines{%
  \endgre@style@normalstafflines%
  \endgroup%
}%

\def\gre@style@lowchoralsign{%
  \begingroup%
  \relax%
}%
\def\endgre@style@lowchoralsign{\endgroup}%

\def\gre@style@highchoralsign{%
  \begingroup%
  \relax%
}%
\def\endgre@style@highchoralsign{\endgroup}%

\def\gre@style@firstsyllableinitial{%
  \begingroup%
  \relax%
}%
\def\endgre@style@firstsyllableinitial{\endgroup}%

\def\gre@style@firstsyllable{%
  \begingroup%
  \relax%
}%
\def\endgre@style@firstsyllable{\endgroup}%

\def\gre@style@firstword{%
  \begingroup%
  \relax%
}%
\def\endgre@style@firstword{\endgroup}%

\def\gre@style@modeline{%
  \begingroup%
  \bf%
  \sc%
  \relax%
}%
\def\endgre@style@modeline{\endgroup}%

\def\gre@style@nabc{%
  \begingroup%
  \relax% we do nothing since colors are implemented in PlainTeX
}
\def\endgre@style@nabc{\endgroup}%

\def\gre@changestyle#1#2[#3]{%
  \gdef\csname gre@style@#1\endcsname{\begingroup#2}%
  \gdef\csname endgre@style@#1\endcsname{#3\endgroup}%
}%

\catcode`\@=\greoldcatcode
