% !TEX root = GregorioRef.tex
% !TEX program = LuaLaTeX
\section{Gregorio Controls}

These functions are the ones written by Gregorio to the gtex file.
While one could, in theory, use/change them to alter the appearance of
elements of the score, it is far better to make your changes in the
gabc file and let Gregorio make the changes to the gtex file.

\macroname{\textbackslash grefirstlinebottomspace}{\#1\#2}{gregoriotex-spaces.tex}
Macro for calculating the additional space needed below the first line of the score.

\begin{argtable}
  \#1 & 0 & no notes below the staff\\
  & 1 & note below the first staff line (c position)\\
  & 2 & note on the 0 staff line (b position)\\
  & 3 & note below the 0 staff line (a position)\\
%  & 4 & note below the 0 staff line with a vertical episemus attached\\
  \#2 & 1 & there is a translation below the text\\
  & 0 & there is no translation below the text
\end{argtable}

%%% Local Variables:
%%% mode: latex
%%% TeX-master: "GregorioRef"
%%% End:
