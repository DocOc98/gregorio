% !TEX root = GregorioRef.tex
% !TEX program = LuaLaTeX+se
\section{Gregorio\TeX{} Controls}

These functions are the ones used by Gregorio\TeX{} internally as it
process the commands listed above.  They should not appear in any user
document and are listed here for programmer documentation purposes
only.

\macroname{\textbackslash greerror}{\#1}{gregoriotex.sty \textup{and} gregoriotex.tex}
Prints an error to the \TeX\ output log.

\begin{argtable}
  \#1 & string & error message\\
\end{argtable}


\macroname{\textbackslash gre@warning}{\#1}{gregoriotex.sty \textup{and} gregoriotex.tex}
Prints a warning to the \TeX\ output log.

\begin{argtable}
  \#1 & string & warning message\\
\end{argtable}

\macroname{\textbackslash gre@metapost}{\#1}{gregoriotex.sty \textup{and} gregoriotex.tex}
Executes \MP{} commands using luamplib.

\begin{argtable}
  \#1 & \MP{} commands & The \MP{} commands to execute.
\end{argtable}

\macroname{\textbackslash gre@deprecated}{\#1\#2}{gregoriotex-main.tex}
Macro that handles deprecated macros. By default, deprecated macros
are allowed and a warning is printed. If the package option
\texttt{deprecated=false} is set, then deprecated macros raise a
package error, halting \TeX.

\begin{argtable}
  \#1 & string & name of the deprecated macro\\
  \#2 & string & name of the correct macro to use\\
\end{argtable}

\macroname{\textbackslash gre@loadgregoriofont}{}{gregoriotex-main.tex}
Loads the chosen font for the neumes at the correct size.

\macroname{\textbackslash gre@calculate@constantglyphraise}{}{gregoriotex-spaces.tex}
%\verb=\gre@calculateconstantglyphraise=%tex.tex
Macro to caluclate \verb=\gre@constantglyphraise=

\macroname{\textbackslash gre@addtranslationspace}{}{gregoriotex-spaces.tex}
%\verb=\gre@addtranslationspace=%tex.tex
Macro to tell Gregorio to set space for the translation.

\macroname{\textbackslash gre@removetranslationspace}{}{gregoriotexspaces.tex}
%\verb=\gre@removetranslationspace=%tex.tex
Macro to tell Gregorio to remove the space allocated to the translation.


\macroname{\textbackslash gre@calculate@additionalspaces}{\#1\#2\#3\#4}{gregoriotex-spaces.tex}
%\verb=\gre@updateadditionalspaces#1#2=%tex.tex
Macro which calculates \verb=\gre@additionalbottomspace= and\\
\verb=\gre@additionaltopspace=

\begin{argtable}
  \#1 & integer & the height number of the top pitch, including signs\\
  \#2 & integer & the height number of the bottom pitch, including signs\\
  \#3 & 0 & there is no translation line\\
      & 1 & there is a translation line\\
  \#4 & 0 & there is no above lines text\\
      & 1 & there is above lines text
 \end{argtable}

\macroname{\textbackslash gre@calculate@textaligncenter}{\#1\#2\#3}{gregoriotex-spaces.tex}
%\verb=\gre@findtextaligncenter#1#2#3=%tex.tex
Macro for calculating \verb=\gre@textaligncenter=.

\begin{argtable}
  \#1 & string & The first part of the syllable (any preceding consonants in Latin).\\
  \#2 & string & The middle part of the syllable (the vowel in Latin, the whole syllable in English).\\
  \#3 & 0 & Calculation is being performed for the current syllable.\\
      & 1 & Calculation is being performed for the next syllable.\\
\end{argtable}

\macroname{\textbackslash gre@calculate@aboveinitialraise}{}{gregoriotex-main.tex}
%\verb=\gre@setaboveinitialrais=%tex.tex
Macro to give \verb=\gre@aboveinitialfirstraise= and\\
\verb=\gre@aboveinitialsecondraise= their working values.

\macroname{\textbackslash gre@englishcentering}{}{gregoriotex-syllable.tex}
Count to track if we are using english centering scheme (1), or not (0).

\macroname{\textbackslash gre@calculate@textlower}{}{gregoriotex-spaces.tex}
Calculates the default value of \texttt{textlower}.  Default is \texttt{spacebeneathtext}.

\macroname{\textbackslash gre@calculate@linewidth}{}{gregoriotex-spaces.tex}
Calculates the default line width.  Default is the width of the printable space (\verb=\hsize=).

\macroname{\textbackslash gre@calculate@stafflinewidth}{}{gregoriotex-spaces.tex}
Calculates the default width of the staff lines.  Default is \texttt{linewidth}.

\macroname{\textbackslash gre@calculate@stafflineheight}{}{gregoriotex-spaces.tex}
Calculates the height (thickness) of the staff lines.  Dependent on \texttt{stafflineheightfactor} and \texttt{grefactor}.

\macroname{\textbackslash gre@calculate@interstafflinespace}{}{gregoriotex-spaces.tex}
Calculates the distance between the staff lines.  Dependent on \texttt{stafflineheight} and \texttt{grefactor}

\macroname{\textbackslash gre@calculate@stafflinediff}{}{gregoriotex-spaces.tex}
Calculates a correction factor for when the staff lines are not their default thickness.  Dependent on \texttt{stafflineheight} and \texttt{grefactor}.

\macroname{\textbackslash gre@calculate@staffheight}{}{gregoriotex-spaces.tex}
Calculates the total height of the staff.  Dependent on \texttt{stafflineheight} and \texttt{interstafflinespace}.

\macroname{\textbackslash gre@calculate@constantglyphraise}{}{gregoriotex-spaces.tex}
Calculates the baseline correction for the glyphs.  Dependent on \texttt{grefactor}, \texttt{additionalbottomspace}, \texttt{spacebeneathtext}, \texttt{spacelinestext}, \texttt{interstafflinespace}, \texttt{stafflineheight}, \texttt{currenttranslationheight}, and \texttt{stafflinediff}.

\macroname{\textbackslash gre@computespaces}{}{gregoriotex-spaces.tex}
Aggregates all of the global distance calculations and calls them in the order needed to respect dependencies.

\macroname{\textbackslash gre@calculate@glyphraisevalue}{\#1\#2}{gregoriotex-spaces.tex}
Calculates the raise values for a glyph (glyphraisevalue and addedraisevalue) based on where it is to be placed and what kind of a glyph it is.  This is a time of use calculation.

\begin{argtable}
  \#1 & integer & The number for where the glyph is located.  \texttt{a} in gabc is 1, \texttt{b} is 2, \etc\\
  \#2 & 0 & no modification\\
  & 1 & puts the value on the interline just above if it is on a line\\
  & 2 & puts the value on the interline just beneath if it is on a line\\
  & 3 & case of the vertical episemus, which is not placed at the same place if the corresponding note is on a line or not\\
  & 4 & case of the punctum mora, for the same reason\\
  & 5 & case of the horizontal episemus under a note, that must be placed a bit lower if the note is on a line\\
  & 6 & case of the signs above (accentus, \etc)\\
  & 8 & case of the punctum mora of the first note of a podatus or the 2nd note of a porrectus, \etc\\
  & 9 & case of the horizontal episemus, that must be placed a bit lower if the note is on a line\\
  & 10 & case of the low choral sign\\
  & 11 & case of the high choral sign\\
  & 12 & case of the low choral sign which is lower than the note\\
  & 13 & case of the brace above the bars
\end{argtable}

\macroname{\textbackslash gre@stafflinefactor}{}{gregoriotex-spaces.tex}
A number indicating the thickness of the staff lines.

\macroname{\textbackslash gre@scale@stafflinefactor}{}{gregoriotex-spaces.tex}
Flag indicating whether the stafflinefactor should scale with changes of grefactor (1), or not (0).

\macroname{\textbackslash gre@calculate@textaligncenter}{\#1\#2\#3}{gregoriotex-spaces.tex}
Macro to calculate the distance from the beginning of the text of a syllable to its alignment point (the center of the vowel for Latin centering, the center of the syllable for English centering).  This is a time of use calculation.

\begin{argtable}
  \#1 & string & the first part of the syllable\\
  \#2 & string & the middle part of the syllable\\
  \#3 & 0 & perform this calculation for the current syllable\\
  & 1 & perform this calculation for the next syllable
\end{argtable}

\macroname{\textbackslash gre@calculate@enddifference}{\#1\#2\#3\#4\#5}{gregoriotex-spaces.tex}
Calculates the difference between the end of the notes and the end of the syllable text.  Also stores the value for the previous syllable if needed.  This is a time of use calculation.

\begin{argtable}
  \#1 & length & the total width of the notes\\
  \#2 & length & the total width of the syllable text\\
  \#3 & length & the alignment distance for the text (\texttt{textaligncenter})\\
  \#4 & length & the alignment distance for the notes (\texttt{notesaligncenter})\\
  \#5 & 0 & do not save the value for the previous syllable before calculating the new value\\
  & 1 & save the value for the previous syllable before calculating the new value
\end{argtable}

\macroname{\textbackslash gre@changeonedimenfactor}{\#1\#2\#3}{gregoriotex-spaces.tex}
Change the scale of a single distance from one factor to another.

\begin{argtable}
  \#1 & string & name of the distance to be scaled.  See \nameref{distances}.\\
  \#2 & integer & the factor the distance is currently in\\
  \#3 & integer & the factor the distance is to be put into\\
\end{argtable}

\macroname{\textbackslash gre@changedimenfactor}{\#1\#2}{gregoriotex-spaces.tex}
Rescales all the distances (and stafflinefactor) which are supposed to scale with a change in factor.

\begin{argtable}
  \#1 & integer & the factor the distance is currently in\\
  \#2 & integer & the factor the distance is to be put into\\
\end{argtable}

\macroname{\textbackslash gre@calculate@nextbegindifference}{\#1\#2\#3\#4}{gregoriotex-spaces.tex}
Macro to calculate \texttt{nextbegindifference}.

\begin{argtable}
  \#1 & string & the first letters of the next syllable\\
  \#2 & string & the middle letters of the next syllable (the vowel in Latin, the whole syllable in English)\\
  \#3 & string & the end letters of the next syllable\\
  \#4 & $0 <$ integer $< 19$ & the type of notes alignment.  See \nameref{notesalign}.\\
  & $20 <$ integer $< 39$ & Same as below 20 except there is a flat before the notes.  Subtract 20 to get the type of notes alignment.\\
  & $40 <$ integer $< 59$ & Same as below 20 except there is a natural before the notes.  Subtract 40 to get the type of notes alignment.
\end{argtable}

\macroname{\textbackslash gre@tempdimcount}{}{gregoriotex-spaces.tex}
Temporary count used in calculations.

\macroname{\textbackslash gre@makein}{\#1}{gregoriotex-spaces.tex}
Strips the decimals and units from a distance.

\begin{argtable}
  \#1 & distance & should be in the form ``[0-9]+.[0-9]+pt''. (\ie the result of applying \verb=\the= to a distance register)
\end{argtable}

\macroname{\textbackslash gre@makenum}{\#1}{gregoriotex-spaces.tex}
Strips the units from a distance. 

\begin{argtable}
  \#1 & distance & should be in the form ``[0-9]+.[0-9]+pt''. (\ie the result of applying \verb=\the= to a distance register)
\end{argtable}

\macroname{\textbackslash gre@unitfactor}{}{gregoriotex-spaces.tex}
Temporary count used by \verb=\gre@convertto=.

\macroname{\textbackslash gre@basefactor}{}{gregoriotex-spaces.tex}
Temporary count used by \verb=\gre@convertto=.

\macroname{\textbackslash gre@temp@count@one}{}{gregoriotex-spaces.tex}
Temporary count used by \verb=\gre@convertto=.

\macroname{\textbackslash gre@convertto}{\#1\#2}{gregoriotex-spaces.tex}
Macro which converts a distance into a particular set of units.  Result is placed in \verb=\gre@converted= as a string.

\begin{argtable}
  \#1 & string & two letter abbreviation for the units.  Should recognize all legal TeX units.\\
  \#2 & distance & Distance to be converted.
\end{argtable}

\macroname{\textbackslash gre@converted}{}{gregoriotex-spaces.tex}
Macro holding result of last call to \verb=\gre@convertto=.

\macroname{\textbackslash gre@consistentunits}{\#1\#2}{gregoriotex-spaces.tex}
This function takes a distance and formats it as a string so that its units conform to the pattern set by a string representation of a distance.  Result is placed in \verb=\gre@stringdist=.

\begin{argtable}
  \#1 & string & the standard whose format is to be matched.\\
  \#2 & distance & the distance to be adjusted.
\end{argtable}

\macroname{\textbackslash gre@stringdist}{}{gregoriotex-spaces.tex}
Macro holding result of last call to \verb=\gre@consistentunits=.

\macroname{\textbackslash gre@temp@count@two}{}{gregoriotex-spaces.tex}
Temporary count used by \verb=\gre@changedimenfactor=.

\macroname{\textbackslash gre@includescore}{\#1}{gregoriotex-main.tex}
Macro that handles \verb=\includescore= calls when they do not have an
optional argument.

\begin{argtable}
  \#1 & string & Relative or absolute path to the score.\\
\end{argtable}

\macroname{\textbackslash gre@includescorewithoption}{[\#1]\#2}{gregoriotex-main.tex}
Macro that handles \verb=\includescore= calls when they have an optional
argument.

\begin{argtable}
  \#1 & \texttt{n} & Optional. \#2 will be included as is. \\
      & \texttt{a} & Optional. Gregoriotex will automatically compile gabc files if necessary.\\
      & \texttt{f} & Optional. Forces gregoriotex to compile the gabc file.\\
  \#2 & string & Relative or absolute path to the score.\\
\end{argtable}

\macroname{\textbackslash gre@writemode}{\#1}{gregoriotex-main.tex}
Macro that writes its argument with \verb=\greannotation=. The
argument typically is given to this macro by \verb=\GreMode= in the
gtex file.

\begin{argtable}
  \#1 & string & Text to place above the initial of a score.\\
\end{argtable}

\macroname{\textbackslash gre@brace@common}{\#1\#2\#3\#4\#5\#6\#7}{gregoriotex-signs.tex}
Common macro used internally to render braces.

\begin{argtable}
  \#1 & length  & The width of the brace.\\
  \#2 & length  & A vertical shift.\\
  \#3 & length  & A horizontal shift.\\
  \#4 & 0       & Don't shift before starting the brace.\\
      & 1       & Shift back a punctum's width before starting the brace.\\
  \#5 & 0       & No accentus above the brace.\\
      & 1       & Typeset an accentus above the brace.\\
  \#6 & integer & The height number for the brace.\\
  \#7 & csname  & The control sequence name representing the brace.
\end{argtable}

\macroname{\textbackslash grebracemetapostpreamble}{\#1}{gregoriotex-signs.tex}
Returns the \MP{} preamble for braces.  The control sequence name does
not have the \texttt{@} symbol because this macro is used within \MP{}.

\begin{argtable}
  \#1 & string & the width of the brace; if \texttt{*}, use the bar brace width.
\end{argtable}

\macroname{\textbackslash gre@draw@curlybrace}{\#1}{gregoriotex-signs.tex}
Draws a curly over-brace using \MP{}.

\begin{argtable}
  \#1 & length & the width of the brace.
\end{argtable}

\macroname{\textbackslash gre@draw@brace}{\#1}{gregoriotex-signs.tex}
Draws a round over-brace using \MP{}.

\begin{argtable}
  \#1 & string & the width of the brace; if \texttt{*}, use the bar brace width.
\end{argtable}

\macroname{\textbackslash gre@draw@underbrace}{\#1}{gregoriotex-signs.tex}
Draws a round under-brace using \MP{}.

\begin{argtable}
  \#1 & length & the width of the brace.
\end{argtable}

\macroname{\textbackslash gre@draw@roundbrace}{\#1}{gregoriotex-signs.tex}
Draws a round over- or under-brace using \MP{}.

\begin{argtable}
  \#1 & length         & the width of the brace.\\
  \#2 & number         & the height of the bounding box in em-relative units.\\
  \#3 & \MP{} commands & \MP{} commands to draw the brace outline.
\end{argtable}

\subsection{Flags}

Flags are either boolean (defined with \verb=\newif=) or macros which expand to an integer.  They store settings and/or the current state of something so that GregorioTeX can typeset things in the desired manner.

All distances in \nameref{distances} have a flag associated with them, of the form \verb=\gre@scale@*=.  This flag
indicates if the distance should scale when the staff size changes (1)
or not (0).

\macroname{\textbackslash ifchecklength}{}{gregoriotex-spaces.tex}
Boolean flag used in \verb=\gresetdim= to indicate if we are attempting to set a rubber length.

\macroname{\textbackslash ifbadlength}{}{gregoriotex-spaces.tex}
Boolean flag used in \verb=\gresetdim= to indicate that we are attempting to assign a rubber length to a distance which cannot accept a rubber value.

\macroname{\textbackslash ifrubber}{}{gregoriotex-spaces.tex}
Boolean flag used in \verb=\gre@changeonedimenfactor= to indicate if we are dealing with one of the distances which can accept a rubber length.

\macroname{\textbackslash ifgre@vowelcentering}{}{gregoriotex-syllable.tex}
Boolean flag used to specify whether the center of the vowel or the center of the syllable is used to align the lyrics with their neumes.

\macroname{\textbackslash ifgre@translationcentering}{}{gregoriotex-main.tex}
Boolean flag used to specify whether the translation text should be centered below its respective syllable.

\macroname{\textbackslash ifgre@removelines}{}{gregoriotex-main.tex}
Boolean flag used to specify whether the staff lines should be removed from the score output.

\macroname{\textbackslash ifgre@hidepclines}{}{gregoriotex-signs.tex}
Boolean flag used to specify whether the staff lines behind a punctum cavum should be hidden.

\macroname{\textbackslash ifgre@hidealtlines}{}{gregoriotex-signs.tex}
Boolean flag used to specify whether the staff lines behind an alteration should be hidden.

\macroname{\textbackslash ifgre@drawbraces}{}{gregoriotex-signs.tex}
Boolean flag used to specify whether braces should be drawn by \MP{} as
opposed to rendered via the score font.


\subsection{Boxes}

Boxes are used to store elements of the score before they are printed for the purposes of reusing them and/or measuring them in order to determine their appropriate placement.

\macroname{\textbackslash gre@box@hep}{}{gregoriotex-chars.tex}
Box for horizontal episemi.

\macroname{\textbackslash gre@temp@width}{}{gregoriotex-main.tex}
Box for holding an element in order to determine its width.

\macroname{\textbackslash gre@box@initial}{}{gregoriotex-main.tex}
Box which holds the initial of the score.

\macroname{\textbackslash gre@box@firstaboveinitial}{}{gregoriotex-main.tex}
Box holding the first (top) annotation above the initial.

\macroname{\textbackslash gre@box@secondaboveinitial}{}{gregoriotex-main.tex}
Box holding the second (bottom) annotation above the initial.

\macroname{\textbackslash gre@box@lines}{}{gregoriotex-main.tex}
Box holding the staff lines.

\macroname{\textbackslash gre@box@temp@sign}{}{gregoriotex-signs.tex}
Box to hold a sign so we can measure it for placement.

\macroname{\textbackslash gre@box@syllablenotes}{}{gregoriotex-syllable.tex}
Box holding the notes associated with a syllable.

\macroname{\textbackslash gre@box@syllabletext}{}{gregoriotex-syllable.tex}
Box holding the text associated with a syllable.



\subsection{Distances}
All of the distances listed in \nameref{distances} have an internal
associated with them, of the form of \verb=\gre@*=, which stores the value of the distance (in
string representation).

These additional distances are calculated by Gregorio based on the values for the user customizable distances and what may be going on in the score at the time of their use.

\macroname{\textbackslash gre@clefwidth}{}{gregoriotex-spaces.tex}
Width of the clef.

\macroname{\textbackslash gre@tempdimsignwidth}{}{gregoriotex-spaces.tex}
Temporary dimension used to calculate placement of signs.

\macroname{gre@tempdimtwo}{}{gregoriotex-spaces.tex}
A temporary dimension used in calculations.

\macroname{\textbackslash gre@constantglyphraise}{}{gregoriotex-spaces.tex}
Dimension representing the space between the 0 of the gregorian fonts and the effective 0 of the TeX score.

\macroname{\textbackslash gre@currenttranslationheight}{}{gregoriotex-spaces.tex}
Dimension representing the space for the translation beneath the text.

\macroname{\textbackslash gre@stafflinewidth}{}{gregoriotex-spaces.tex}
Dimension representing the width of a line of staff.  Can vary, for
example, at the first line.

\macroname{\textbackslash gre@linewidth}{}{gregoriotex-spaces.tex}
Dimension representing the width of the score (including initial).

\macroname{\textbackslash gre@additionalbottomspace}{}{gregoriotex-spaces.tex}
Dimension representing extra space below the staff needed for low notes.

\macroname{\textbackslash gre@additionaltopspace}{}{gregoriotex-spaces.tex}
Dimension representing extra space above the staff needed for high notes.

\macroname{\textbackslash gre@textlower}{}{gregoriotex-spaces.tex}
Dimension representing the height of the separation between the 0th
line (which is invisible except for notes in the a or b position) and
the bottom of the text.

\macroname{\textbackslash gre@tempwidth}{}{gregoriotex-spaces.tex}
Dimension representing width of some element.

\macroname{\textbackslash gre@textaligncenter}{}{gregoriotex-spaces.tex}
Dimension representing the width from the beginning of the letters in
a syllable to the middle of the middle letters.  Used for lining up
neumes and syllables.

\macroname{\textbackslash gre@additionalleftspace}{}{gregoriotex-spaces.tex}
Dimension representing the additional space that has to be added to
the localleftbox for a big initial (one taking two lines).

\macroname{\textbackslash gre@initialwidth}{}{gregoriotex-spaces.tex}
Dimension representing the width of the initial (and the space after).

\macroname{\textbackslash gre@aboveinitialfirstraise}{}{gregoriotex-spaces.tex}
Dimension representing the space allocated to the first annotation.

\macroname{\textbackslash gre@aboveinitialsecondraise}{}{gregoriotex-spaces.tex}
Dimension representing the space allocated to the second annotation.

\macroname{\textbackslash gre@currentabovelinestextheight}{}{gregoriotex-spaces.tex}
Dimension representing the space allocated above the lines for text.	

\macroname{\textbackslash gre@staffheight}{}{gregoriotex-spaces.tex}
The total height of the staff including the width of the lines and the spaces between them.

\macroname{\textbackslash gre@stafflinediff}{}{gregoriotex-spaces.tex}
Distance representing the difference between the actual size of the staff lines and the ``standard'' size.

\macroname{\textbackslash gre@stafflineheight}{}{gregoriotex-spaces.tex}
The height of the staff line.

\macroname{\textbackslash gre@interstafflinespace}{}{gregoriotex-spaces.tex}
The space between the lines.

\macroname{\textbackslash gre@glyphraisevalue}{}{gregoriotex-spaces.tex}
The value that a particular glyph must be raised to be set in the correct position.

\macroname{\textbackslash gre@addedraisevalue}{}{gregoriotex-spaces.tex}
The additional raise needed for the vertical episema and the puncta.

\macroname{\textbackslash gre@enddifference}{}{gregoriotex-spaces.tex}
Distance from the end of the notes to the end of the text for the previous syllable.  Positive values when notes go further than text, negative in the other case.

\macroname{\textbackslash gre@previousenddifference}{}{gregoriotex-spaces.tex}
Stored value of enddifference prior to the current one.

\macroname{\textbackslash gre@nextbegindifference}{}{gregoriotex-spaces.tex}
The difference between the start of the notes and the start of the text for the next syllable.  Positive when when text begins first, negative in other case.

\macroname{\textbackslash gre@begindifference}{}{gregoriotex-spaces.tex}
The difference between the start of the notes and the start of the text for the current syllable.  Positive when when text begins first, negative in other case.

\macroname{\textbackslash gre@lastglyphwidth}{}{gregoriotex-spaces.tex}
The width of the last glyph.

\macroname{\textbackslash gre@notesaligncenter}{}{gregoriotex-spaces.tex}
Distance from beginning of notes to their point of alignment.

\macroname{\textbackslash gre@tempdim}{}{gregoriotex-spaces.tex}
Temporary dimension used in calculations.

\macroname{\textbackslash gre@tempdimskip}{}{gregoriotex-spaces.tex}
Temporary skip used in calculations.

\macroname{\textbackslash gre@skipone}{}{gregoriotex-spaces.tex}
Temporary skip used in calculations.

\macroname{\textbackslash gre@temp}{}{gregoriotex-spaces.tex}
Temporary skip used in calculations.

\macroname{\textbackslash gre@temp@dimen@one}{}{gregoriotex-spaces.tex}
Temporary dimension used in calculations.

\macroname{\textbackslash gre@temp@skip@one}{}{gregoriotex-spaces.tex}
Temporary skip used in calculations.

\macroname{\textbackslash gre@unit}{}{gregoriotex-spaces.tex}
Temporary dimension used by \verb=\gre@convertto=.

\macroname{\textbackslash gre@base}{}{gregoriotex-spaces.tex}
Temporary dimension used by \verb=\gre@convertto=.

\macroname{\textbackslash gre@maxlen}{}{gregoriotex-spaces.tex}
Distance holding the maximum legal length in TeX.



\section{Special arguments}

These arguments are used by multiple functions and take a lot of space
to describe so we describe them once here and refer to this section
rather than have multiple definitions.

\subsection{Note Alignment Type}\label{notesalign}
\rowcolors{1}{lightgray}{lightgray}
\begin{tabular}{cp{10.5cm plus .5cm}}
  \multicolumn{2}{c}{Integer with the following possibilities:} \\
  \hline
  0 & one-note glyph or more than two notes glyph except porrectus : here we must put the aligncenter in the middle of the first note\\
  1 & two notes glyph (podatus is considered as a one-note glyph) : here we put the aligncenter in the middle of the glyph\\
  2 & porrectus : has a special align center\\
  3 & initio-debilis : same as 1 but the first note is much smaller\\
  4 & case of a glyph starting with a quilisma\\
  5 & case of a glyph starting with an oriscus\\
  6 & case of a punctum inclinatum\\
  7 & case of a stropha\\
  8 & flexus with an ambitus of one\\
  9 & flexus deminutus\\
  10 & virgula\\
  11 & divisio minima, minor and maior\\
  12 & divisio finalis
 \end{tabular}

\subsection{Episemus Special}\label{EpisemusSpecial}
\definecolor{shadecolor}{named}{lightgray}%
\begin{shaded*}%
\vspace{-1.4\baselineskip}
\begin{center}String with the following possibilities:\end{center}
\vspace{-0.8\baselineskip}
\hrule
\vspace{-0.8\baselineskip}
\begin{description}
  \item[FinalPunctum] Last note, which is a standard punctum (works with pes).
  \item[FinalDeminutus] Same, but the last note is a deminutus.
  \item[PenultBeforePunctumWide] The note before the last note, which is a standard punctum.
  \item[PenultBeforeDeminutus] Idem, but the note is the note preceding a deminutus.
  \item[AntepenultBeforePunctum] The note before the note before the last note (for porrectus flexus).
  \item[AntepenultBeforeDeminutus] Idem, but when the two last notes are a deminutus.
  \item[InitialPunctum] The first note, if it is a standard punctum.
  \item[InitioDebilis] The first note, if it is an initio debilis.
  \item[PorrNonAuctusInitialWide] first note of a non-auctus porrectus with a second ambitus of at least two.
  \item[PorrNonAuctusInitialOne] first note of a non-auctus porrectus with a second ambitus of one
  \item[PorrAuctusInitialAny] first note of an auctus porrectus, regardless of second ambitus
  \item[FinalInclinatum] punctum inclinatum as last note
  \item[FinalInclinatumDeminutus] punctum inclinatum deminutus as last note
  \item[FinalStropha] stropha as last note
  \item[FinalQuilisma] quilisma as last note
  \item[FinalOriscus] oriscus as last note
  \item[PenultBeforePunctumOne] second-to-last note, with a second ambitus of one, when last note is a standard punctum (like the second note of ghg)
  \item[FinalUpperPunctum] "upper smaller punctum" as last note (concerning simple podatus, podatus, and torculus resupinus)
  \item[InitialOriscus] oriscus as first note, disconnected from next note
  \item[InitialQuilisma] quilisma as first note, disconnected from next note
  \item[TorcResNonAuctusSecondWideWide] second note of a non-auctus torculus resupinus starting with a punctum, with a first and second ambitus of at least two
  \item[TorcResNonAuctusSecondOneWide] second note of a non-auctus torculus resupinus starting with a punctum, with a first ambitus of one and a second ambitus of at least two
  \item[TorcResDebilisNonAuctusSecondAnyWide] second note of a non-auctus torculus resupinus initio debilis with any first ambitus and a second ambitus of at least two
  \item[FinalLineaPunctum] linea punctum (cavum) as last note
  \item[BarStandard] standard bar
  \item[BarVirgula] virgula
  \item[BarDivisioFinalis] divisio finalis
  \item[TorcResQuilismaNonAuctusSecondWideWide] second note of a non-auctus torculus resupinus starting with a quilisma, with a first and second ambitus of at least two
  \item[TorcResOriscusNonAuctusSecondWideWide] second note of a non-auctus torculus resupinus starting with an oriscus, with a first and second ambitus of at least two
  \item[TorcResQuilismaNonAuctusSecondOneWide] second note of a non-auctus torculus resupinus starting with a quilisma, with a first ambitus of one and and second ambitus of at least two
  \item[TorcResOriscusNonAuctusSecondOneWide] second note of a non-auctus torculus resupinus starting with an oriscus, with a first ambitus of one and and second ambitus of at least two
  \item[TorcResNonAuctusSecondWideOne] second note of a non-auctus torculus resupinus starting with a punctum, with a first ambitus of at least two and a second ambitus of one
  \item[TorcResDebilisNonAuctusSecondAnyOne] second note of a non-auctus torculus resupinus initio debilis with any first ambitus and a second ambitus of one
  \item[TorcResQuilismaNonAuctusSecondWideOne] second note of a non-auctus torculus resupinus starting with a quilisma, with a first ambitus of at least two and a second ambitus of one
  \item[TorcResOriscusNonAuctusSecondWideOne] second note of a non-auctus torculus resupinus starting with an oriscus, with a first ambitus of at least two and a second ambitus of one
  \item[TorcResNonAuctusSecondOneOne] second note of a non-auctus torculus resupinus starting with a punctum, with a first and second ambitus of one
  \item[TorcResQuilismaNonAuctusSecondOneOne] second note of a non-auctus torculus resupinus starting with a quilisma, with a first and second ambitus of one
  \item[TorcResOriscusNonAuctusSecondOneOne] second note of a non-auctus torculus resupinus starting with an oriscus, with a first and second ambitus of one
  \item[TorcResAuctusSecondWideAny] second note of an auctus torculus resupinus starting with a punctum, with a first ambitus of at least two and any second ambitus
  \item[TorcResDebilisAuctusSecondAnyAny] second note of an auctus torculus resupinus initio debilis with any first and second ambitus
  \item[TorcResQuilismaAuctusSecondWideAny] second note of an auctus torculus resupinus starting with a quilisma, with a first ambitus of at least two and any second ambitus
  \item[TorcResOriscusAuctusSecondWideAny] second note of an auctus torculus resupinus starting with an oriscus, with a first ambitus of at least two and any second ambitus
  \item[TorcResAuctusSecondOneAny] second note of an auctus torculus resupinus starting with a punctum, with a first ambitus of one and any second ambitus
  \item[TorcResQuilismaAuctusSecondOneAny] second note of an auctus torculus resupinus starting with a quilisma, with a first ambitus of one and any second ambitus
  \item[TorcResOriscusAuctusSecondOneAny] second note of an auctus torculus resupinus starting with an oriscus, with a first ambitus of one and any second ambitus
  \item[ConnectedPenultBeforePunctumWide] second-to-last note connected to prior note, with a second ambitus of at least two, when last note is a standard punctum (like the second note of \textit{gig})
  \item[ConnectedPenultBeforePunctumOne] second-to-last note connected to prior note, with a second ambitus of one, when last note is a standard punctum (like the second note of \textit{gih})
  \item[InitialConnectedPunctum] standard punctum as first note, connected to next higher note
  \item[InitialConnectedVirga] "virga" as first note, connected to next lower note
  \item[InitialConnectedQuilisma] quilisma as first note, connected to next higher note
  \item[InitialConnectedOriscus] oriscus as first note, connected to next higher note
  \item[FinalConnectedPunctum] punctum as last note, connected to prior higher note
  \item[FinalConnectedAuctus] auctus as last note, connected to prior lower note
  \item[FinalVirgaAuctus] virga aucta as last note
  \item[FinalConnectedVirga] "virga" as last note, connected to prior lower note
  \item[InitialVirga] "virga" as first note, disconnected from next note
\end{description}
\end{shaded*}

%%% Local Variables:
%%% mode: latex
%%% TeX-master: "GregorioRef"
%%% End:
