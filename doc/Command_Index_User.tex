% !TEX root = GregorioRef.tex
% !TEX program = LuaLaTeX
\section{User Controls}

These functions are available to the user to customize elements of the
score which cannot be controlled from the gabc file. They can be added
to any \verb=.tex= file. Do not add them to any \verb=.gtex= or
\verb=.gabc= file.

\subsection{Using the Package}

To use the gregoriotex package include \verb=\usepackage{gregoriotex}=
in the document preamble. This macro has the following form:

\macroname{\textbackslash usepackage}{[\optional{(options)}]\{gregoriotex\}}{gregoriotex.sty}

The optional arguments are:

\bigskip\rowcolors{1}{lightgray}{lightgray}
\begin{tabular}{lp{10cm plus .5cm}}
  Argument & Description \\
  \hline
  debug & Debug messages will be printed to the output log. \\
  \hline
  nevercompile & Default. The classic behavior of gregoriotex. The user is %
    responsible for compiling gabc scores into gtex files.\\
  autocompile & Gregoriotex will automatically compile gtex files from gabc %
    files when necessary. If the gabc has been modified, or the %
    gtex has an outdated version, or the gtex file does not exist, %
    THEN gregoriotex will compile a new gtex file.\\
  forcecompile & Gregoriotex will compile all scores from their gabc files.\\
\end{tabular}\bigskip

\textbf{Note:} \verb=nevercompile=, \verb=autocompile=, and
\verb=forcecompile= conflict with eachother. Only one should be
specified in the options list.

\subsection{Commands}

In general, commands should not be modified.  Exceptions are noted below.

\subsubsection{Including scores}

\macroname{\textbackslash includescore}{[\optional{\#1}]\{\#2\}}{gregoriotex-main.tex}
Macro for including scores.  Works on both gabc and tex files.

\begin{argtable}
  \#1 & \texttt{n} & Optional. \#2 will be included as is. \\
      & \texttt{a} & Optional. Gregoriotex will automatically compile gabc files if necessary.\\
      & \texttt{f} & Optional. Forces gregoriotex to compile the gabc file.\\
  \#2 & string & Relative or absolute path to the score.\\
\end{argtable}

Example:\par\medskip
\begin{latexcode}
  \includescore[n]{TecumPrincipium.tex}
  \includescore{Chant/VirgoVirginum.gabc}
  \includescore{/home/user/chant/AdTeLevavi}
  \includescore[a]{AveMaria}

  %The following lines include the same score:
  \includescore{Christus}
  \includescore{Christus.gtex}
  \includescore{./Christus}
  \includescore{./Christus.gabc}

  %With the optional arg [f], #2 must be a file usable by TeX.
  \includescore[f]{TecumPrincipium.gabc} % Wrong
\end{latexcode}

\textbf{Important:} For the sake of clarity it is recommended that the
file extension be omitted from \texttt{\#2}. When the nevercompile
option is in effect (either via package option
\texttt{[nevercompile]}, or \verb=\nevercompilegabc=, or
\verb=\includescore[n]=) \#2 must be a \TeX\ file that exists and
\texttt{.tex} is the only extension that may be omitted.

When called with the optional argument \texttt{[a]} gregoriotex will
automatically generate a \texttt{gtex} file in this format:
\texttt{\textit{scorename}-x\_x\_x.gtex} where \texttt{x\_x\_x} is the
gregorio version. This resulting file is not intended to be modified
by the user and will be removed when the gabc file is recompiled. The
rules that gregoriotex uses to determine if a gabc file needs to be
compiled are:

\begin{itemize}
\item If a gtex file does not exist.
\item If the modification time of the gabc file is newer than its
  corresponding gtex file.
\item If the version of the gtex file is outdated.
\end{itemize}

When called with the optional argument \texttt{[n]} gregoriotex will
include the score without doing anything else. This is the same as the
old behavior of gregoriotex and the default behavior.

When called with the optional argument \texttt{[f]} gregoriotex will
compile the gabc file into a gtex file. This is similar to
\texttt{[a]} except the gabc is compiled every time.

\macroname{\textbackslash forcecompilegabc}{}{gregoriotex-main.tex} A switch
to change the behavior of the way gregoriotex includes scores. When
used, all later calls of \verb=\includescore= will compile the gabc
file into a gtex file. This is similar to using the package option
\verb=[forcecompile]=, but does not necessarly apply to the entire
document.

\macroname{\textbackslash autocompilegabc}{}{gregoriotex-main.tex} A switch
to change the behavior of the way gregoriotex includes scores. When
used, all later calls of \verb=\includescore= will use gregoriotex's
automatic compilation of gabc files. This is similar to using the
packape option \verb=[autocompile]=, but does not necessarly apply to
the entire document.

\macroname{\textbackslash nevercompilegabc}{}{gregoriotex-main.tex} A switch
to change the behavior of the way gregoriotex includes scores. When
used, all later calls of \verb=\includescore= will include the score
without doing anything else. This is the same as the traditional
behavior of gregoriotex. It is the package default.

\medskip The three previous macros can be combined in the same document to
switch between different includescore behaviors: \par\medskip
\begin{latexcode}
  \usepackage{gregoriotex} % [nevercompile] is the default.
  ----
  \includescore{TecumPrincipium} % gabc never compiled.
  \includescore[f]{TecumPrincipium} % gabc always compiled.
  \includescore[a]{TecumPrincipium} % gabc auto compiled.

  \autocompilegabc
  \includescore{TecumPrincipium} % gabc auto compiled.
  \includescore[n]{TecumPrincipium} % gabc never compiled.
  \includescore[f]{TecumPrincipium} % gabc always compiled.

  \forcecompilegabc
  \includescore{TecumPrincipium} % gabc always compiled.
  \includescore[n]{TecumPrincipium} % gabc never compiled.
  \includescore[a]{TecumPrincipium} % gabc auto compiled.
\end{latexcode}

\macroname{\textbackslash gabcsnippet}{[\optional{\#1}]\{\#2\}}{gregoriotex-main.tex}
Converts the gabc notation specified in \texttt{\#2} to GregorioTeX and
includes it directly in the document.  The optional argument \texttt{[\#1]}
may be used to specify a gabc \texttt{initial-style}, which defaults to
\texttt{1}.

\begin{argtable}
  \#1 & number & Optional.  The gabc \texttt{initial-style} value to use.\\
  \#2 & string & The gabc to insert into the document.\\
\end{argtable}

\medskip For example:\par\medskip
\begin{latexcode}
  \gabcsnippet[0]{(c3) Al(eg~)le(gv.fhg)lu(efe___)ia(e.) (::)}
\end{latexcode}

\macroname{\textbackslash greforcehyphen}{}{gregoriotex.tex}
Tell Gregorio\TeX\ to force the appearance of hyphens between all
syllables in polysyllabic words in a score.  This is done by overriding \texttt{maximumspacewithoutdash} so subsequent changes to this dimension will override this command.

\macroname{\textbackslash greautohyphen}{}{gregoriotex.tex}
Tell Gregorio\TeX\ to determine the appearance of hyphens between
syllables in polysyllabic words in a score based on spacing.  This is done by setting \texttt{maximumspacewithoutdash} back to the value found in \texttt{gsp-default.tex}.

\subsubsection{Styling}

Different elements of an include score have different styles applied.  These elements and their defaults are listed below:

\bigskip\rowcolors{1}{lightgray}{lightgray}
\begin{tabular}{lp{10cm plus .5cm}r}
  Element Name & Description & Default\\
  initial & Normal Initials & 40 pt font\\
  biginitial & Big (2-Line) Initials & 80 pt font\\
  translation & Translation text (appears below lyrics) & italics\\
  abovelinestext & Above line text (\texttt{<alt></alt>} in gabc, appears above the staff) & normal\\
  normalstafflines & Full length staff lines & none\\
  additionalstafflines & short lines behind notes above or below the staff & inherit from \texttt{normalstafflines}\\
  lowchoralsign & low choral signs & none\\
  highchoralsign & high choral signs & none\\
  firstsyllableinitial & the first letter of the first syllable of a score which is not the score initial & none\\
  firstsyllable & the balance of the first syllable of the score & none
\end{tabular}

\macroname{\textbackslash grechangestyle}{\{\#1\}\{\#2\}[\#3]}{gregoriotex.sty and gregoriotex.tex}
Command to change styling of a score element.

\begin{argtable}
  \#1 & string & element whose styling is to be changed (see list above for options)\\
  \#2 & \TeX\ code & the code necessary to turn on the styling\\
  \#3 & \TeX\ code & Optional.  The code necessary to turn off the styling (e.g. if the code to turn on the styling contains a \verb=\begin{environment}= then the code to turn it off must have the matching \verb=\end{environment}=.
\end{argtable}

Examples:\par\medskip
\begin{latexcode}
  \grechangestyle{translation}{\it\bf} % Works for both PlainTeX and LaTeX this would make the translations bold and italic
  \grechangestyle{abovelinetext}{\begin{small}\begin{italic}}[\end{italic}\end{small}] % LaTeX only, this would make the above lines text small and italic
  \grechangestyle{initial}{\fontsize{36}{36}\selectfont} % This would make the initial print in 36pt font.
\end{latexcode}

Each element will be typeset within an isolated group to prevent styling commands from leaking from one element to the next.  As a result, if a styling command has an ``on-switch'' but no ``off-switch'' (like \verb=\it= or \verb=\bf= in the first example above) it is not necessary to encapsulate them within a \verb=\begingroup= and \verb=\endgroup=.  As a result, the third argument is only necessary for styling commands which come in pairs (like the environments in the second example).

\subsubsection{Alternate glyph drawings}

\macroname{\textbackslash grechangeglyph}{\{\#1\}\{\#2\}\{\#3\}}{gregoriotex-main.tex}
Substitutes the given GregorioTeX score glyph with the specified glyph
from the specified font.

\begin{argtable}
  \#1 & string & The name of the GregorioTeX glyph to replace.\\
  \#2 & string & The name of the font to use.\\
  \#3 & number & The code point of the glyph to use.\\
  & \texttt{.}string & The name of the variant (appended to \#1) to use.\\
  & string & (any other string) The name of the glyph to use.
\end{argtable}

\medskip If \texttt{\#1} has a wildcard (a \texttt{*}) in it, then
\texttt{\#3} must start with a dot and all glyphs matching \texttt{\#1}
will be replaced with corresponding glyphs whose names have \texttt{\#3}
appended.

\medskip If \texttt{\#2} is \texttt{*}, then the substitution is assumed
to be available in all score fonts.

\medskip For example, to use the old glyphs (from Caeciliae) for the
strophicus, use the following:\par\medskip
\begin{latexcode}
  \grechangeglyph{Stropha}{greciliae}{.caeciliae}
  \grechangeglyph{StrophaAucta}{greciliae}{.caeciliae}
\end{latexcode}

\medskip To replace all torculus resupinus glyphs with their alternate
versions, use the following:\par\medskip
\begin{latexcode}
  \grechangeglyph{TorculusResupinus*}{*}{.alt}
\end{latexcode}

\macroname{\textbackslash greresetglyph}{\{\#1\}}{gregoriotex-main.tex}
Removes a GregorioTeX score glyph substitution, restoring it back to
its original form.

\begin{argtable}
  \#1 & string & The name of the GregorioTeX glyph to restore.\\
\end{argtable}

\medskip If \texttt{\#1} has a wildcard (a \texttt{*}) in it, then
all glyphs matching \texttt{\#1} will be restored.

\medskip For example, to restore the strophicus back to the new glyphs,
use the following:\par\medskip
\begin{latexcode}
  \greresetglyph{Stropha}
  \greresetglyph{StrophaAucta}
\end{latexcode}

\medskip To restore all torculus resupinus glyphs to their original
form, use the following:\par\medskip
\begin{latexcode}
  \greresetglyph{TorculusResupinus*}
\end{latexcode}

\macroname{\textbackslash gredefsymbol}{\{\#1\}\{\#2\}\{\#3\}}{gregoriotex-symbols.tex}
Defines (or redefines) a TeX control sequence to be a non-score symbol.
If defined this way, the symbol will scale with the text font.

\begin{argtable}
  \#1 & string & The name of the TeX control sequence (without leading backslash).\\
  \#2 & string & The name of the font to use.\\
  \#3 & number & The code point of the glyph to use.\\
  & string & The name of the glyph to use.
\end{argtable}

\macroname{\textbackslash gredefsizedsymbol}{\{\#1\}\{\#2\}\{\#3\}}{gregoriotex-symbols.tex}
Defines (or redefines) a TeX control sequence to be a non-score symbol
which requires a single numeric argument (in points) to which the symbol
will be scaled.

\begin{argtable}
  \#1 & string & The name of the TeX control sequence (without leading backslash).\\
  \#2 & string & The name of the font to use.\\
  \#3 & number & The code point of the glyph to use.\\
  & string & The name of the glyph to use.
\end{argtable}

\subsubsection[Barred letters (A/, etc.)]{Barred letters (\Abar, etc.)}

\macroname{\textbackslash gresimpledefbarglyph}{\{\#1\}\{\#2\}}{gregoriotex-symbols.tex}
Redefines a TeX control sequence to be a a barred glyph.

\begin{argtable}
  \#1 & character & must be A, R or V.\\
  \#2 & dimension & how much the bar will be shifted left.\\
\end{argtable}

Gregorio does not have precomposed barred letters, instead, it has bars that you
can you to composed barred letters in your text font. This command is the most
simple version.

For example:

\medskip \begin{latexcode}
  \gresimpledefbarglyph{A}{0.3em}
\end{latexcode}

Will define \texttt{\textbackslash Abar} to be a A with a bar shifted right
of \texttt{0.3em} from the beginning of the glyph. This is the default definition
and fits well with the Linux Libertine font. If you use another font, you'll
certainly have to change this value by calling the \texttt{gresimpledefbarglyph} command.

\macroname{\textbackslash greexpldefbarglyph}{\{\#1\}\{\#2\}\{\#3\}\{\#4\}\{\#5\}\{\#6\}}{gregoriotex-symbols.tex}
Redefines a TeX control sequence to be a a barred glyph.

\begin{argtable}
  \#1 & string & the name of the command you want to define.\\
  \#2 & string & command to typeset the text.\\
  \#3 & string & symbol of the bar (must be defined through \texttt{gredefsizedsymbol}).\\
  \#4 & number & the size of greextra to use (in pt).\\
  \#5 & dimension & horizontal right shift of the bar.\\
  \#6 & dimension & vertical shift of the bar glyph.\\
\end{argtable}

This is a more complete version of the previous command, it allows you to define
barred letters with a different style. For example you can choose another bar
drawing, or take a bar more adapted to small font size.

For example:

\begin{footnotesize}
\begin{latexcode}
  \greexpldefbarglyph{RBarBold}{\textbf{R}}{greRBarSmall}{13}{1.7mm}{0.1mm}
\end{latexcode}
\end{footnotesize}

\greexpldefbarglyph{RBarBold}{\textbf{R}}{greRBarSmall}{13}{1.7mm}{0.1mm}

Will define \texttt{\textbackslash RBarBold} to be a bold \textbf{R} with 
the bar made for small text (a bit bolder, named \texttt{RBarSmall} in greextra)
, at 12pt, shifted right of \texttt{1.7mm} from the beginning of the glyph, and lowered down
by \texttt{0.1mm}. The result is that \texttt{\textbackslash RBarBold} will typeset \RBarBold .

See Annex ?? for a list of bars and other symbols present in the greextra font.

\subsubsection{Text centering}

\macroname{\textbackslash englishcentering}{}{gregoriotex-syllable.tex}
Changes settings to align syllables in the English style (center of
syllable aligns with center of first glyph in neume).  Will only
affect scores which do not have “centering-scheme: english;” in their
gabc header.

\macroname{\textbackslash defaultcentering}{}{gregoriotex-syllable.tex}
Changes settings to align syllables in the Latin style (center of
vowel aligns with center of first glyph in neume).  Will only affect
scores which do not have “centering-scheme: english;” in their gabc
header.

\subsubsection{\emph{Euouae} block}

\macroname{\textbackslash grenobreakineuouae}{}{gregoriotex-syllable.tex}
Prevents line breaking in Euouae blocks.

\macroname{\textbackslash greallowbreakineuouae}{}{gregoriotex-syllable.tex}
Allows line breaking in Euouae blocks.


\subsubsection{Spacings}

\macroname{\textbackslash gresetdim}{\{\#1\}\{\#2\}\{\#3\}}{gregoriotex-spaces.tex}
Macro to set one of gregoriotex’s distances.  Does not update dependent distances.  Used primarily to initialize distances in a space configuration file.  This function if \texttt{gre} prefixed to highlight that it should not be used for distances not defined in Gregorio\TeX.

\begin{argtable}
  \#1 & string & The name of the distance to be changed.  See \nameref{distances} below.\\
  \#2 & string & The distance in string format.  \textbf{Note:} You cannot use a length register for this argument.  You \emph{must} use a string because of the way that gregoriotex handles spaces.\\
  \#3 & 0 & Distance will not scale when staff size is changed.\\
  & 1 & Distance will scale when staff size is changed.
\end{argtable}

\macroname{\textbackslash grechangedim}{\{\#1\}\{\#2\}\{\#3\}}{gregoriotex-spaces.tex}
Macro to change one of gregoriotex’s distances and update any dependent distances.  This function if \texttt{gre} prefixed to highlight that it should not be used for distances not defined in Gregorio\TeX.

\begin{argtable}
  \#1 & string & The name of the distance to be changed.  See \nameref{distances} below.\\
  \#2 & string & The distance in string format.  \textbf{Note:} You cannot use a length register for this argument.  You \emph{must} use a string because of the way that gregoriotex handles spaces.\\
  \#3 & 0 & Distance will not scale when staff size is changed.\\
  & 1 & Distance will scale when staff size is changed.
\end{argtable}

\macroname{\textbackslash grenoscaledim}{\{\#1\}}{gregoriotex-spaces.tex}
Macro to turn off scaling for a particular distance.  This function if \texttt{gre} prefixed to highlight that it should not be used for distances not defined in Gregorio\TeX.

\begin{argtable}
  \#1 & string & The name of the distance for which scaling it to be turned off.  See \nameref{distances} below.
\end{argtable}

\macroname{\textbackslash grescaledim}{\{\#1\}}{gregoriotex-spaces.tex}
Macro to turn on scaling for a particular distance.  This function if \texttt{gre} prefixed to highlight that it should not be used for distances not defined in Gregorio\TeX.

\begin{argtable}
  \#1 & string & The name of the distance for which scaling it to be turned on.  See \nameref{distances} below.
\end{argtable}

\macroname{\textbackslash greloadspaceconf}{\{\#1\}}{gregoriotex-spaces.tex}
Macro to load a space configuration file.  Space configuration file names have the format \verb=gsp-identifier.tex= and must be in the same directory as your project or in your texmf directory.  This function if \texttt{gre} prefixed to highlight that it only loads distances for Gregorio\TeX.

\begin{argtable}
  \#1 & string & The identifier of the space configuration file.
\end{argtable}

Example:\par\medskip
\begin{latexcode}
  % loads gsp-default.tex, the default configuration file
  \GreLoadSpaceConf{default}
  % loads a custom configuration called gsp-myspaces.tex
  \GreLoadSpaceConf{myspaces}
\end{latexcode}

\macroname{\textbackslash setaboveinitialseparation}{\{\#1\}\{\#2\}}{gregoriotex-spaces.tex}
Macro to set the spacing between the annotations.

\begin{argtable}
  \#1 & string & The distance in string format.  \textbf{Note:} You cannot use a length register for this argument.  You \emph{must} use a string because of the way that gregoriotex handles spaces.\\
  \#2 & 0 & Distance will not scale when staff size is changed.\\
  & 1 & Distance will scale when staff size is changed.
\end{argtable}

\textbf{Warning:} This function formerly had only one arguments.  It now takes 2.

Deprecated version: \verb=\GreSetAboveInitialSeparation= which only took the first argument.  If you use the deprecated command the second argument will be assumed to be 1.

\macroname{\textbackslash setspaceafterinitial}{\{\#1\}\{\#2\}}{gregoriotex-spaces.tex}
Macro to set the spacing after the initial (and before the staff lines).

\begin{argtable}
  \#1 & string & The distance in string format.  \textbf{Note:} You cannot use a length register for this argument.  You \emph{must} use a string because of the way that gregoriotex handles spaces.\\
  \#2 & 0 & Distance will not scale when staff size is changed.\\
  & 1 & Distance will scale when staff size is changed.
\end{argtable}

\textbf{Warning:} This function formerly had only one arguments.  It now takes 2.

Deprecated version: \verb=\GreSetSpaceAfterInitial= which only took the first argument.  If you use the deprecated command the second argument will be assumed to be 1.

\macroname{\textbackslash setspacebeforeinitial}{\{\#1\}\{\#2\}}{gregoriotex-spaces.tex}
Macro to set the spacing before the initial.

\begin{argtable}
  \#1 & string & The distance in string format.  \textbf{Note:} You cannot use a length register for this argument.  You \emph{must} use a string because of the way that gregoriotex handles spaces.\\
  \#2 & 0 & Distance will not scale when staff size is changed.\\
  & 1 & Distance will scale when staff size is changed.
\end{argtable}

\textbf{Warning:} This function formerly had only one arguments.  It now takes 2.

Deprecated version: \verb=\GreSetSpaceBeforeInitial= which only took the first argument.  If you use the deprecated command the second argument will be assumed to be 1.

\macroname{\textbackslash setinitialspacing}{\{\#1\}\{\#2\}\{\#3\}\{\#4\}}{gregoriotex-spaces.tex}
Macro to set the spacing around and for the initial.

\begin{argtable}
  \#1 & string & The distance before the initial in string format.  \textbf{Note:} You cannot use a length register for this argument.  You \emph{must} use a string because of the way that gregoriotex handles spaces.\\
  \#2 & string & The distance for the initial in string format.  This distance is ignored if set to 0pt.  \textbf{Note:} You cannot use a length register for this argument.  You \emph{must} use a string because of the way that gregoriotex handles spaces.\\
  \#3 & string & The distance after the initial in string format.  \textbf{Note:} You cannot use a length register for this argument.  You \emph{must} use a string because of the way that gregoriotex handles spaces.\\
  \#4 & 0 & Distance will not scale when staff size is changed.\\
  & 1 & Distance will scale when staff size is changed.
\end{argtable}

\textbf{Warning:} This function formerly had only three arguments.  It now takes 4.

\macroname{\textbackslash setstafflinethickness}{\{\#1\}}{gregoriotex-spaces.tex}
Macro to adjust the thickness of the staff lines.

\begin{argtable}
  \#1 & integer & The relative thickness of the staff lines.  The default value is 10.  Higher numbers yield thicker staff lines.
\end{argtable}

Deprecated version: \verb=\gresetstafflinefactor=.




\subsection{Distances}\label{distances}

Each of the following distances controls some aspect of the spacing of the Gregorio score.  They are changed using commands documented above (\eg \verb=\changedim=).  If the distance permits a rubber value then the default value will indicate the stretch and shrink (even if they are zero by default).  Distances whose default value does not include a stretch or shrink may not take a rubber value.

\emph{Note:} Because of the way Gregorio handles distances these cannot be manipulated as if they were normal \TeX\ dimensions or skips.  As a result they should only be changed using the commands defined by gregoriotex for this purpose.

\macroname{additionallineswidth}{}{gsp-default.tex}
The additional width of the additional lines (compared to the width of the glyph they're associated with).  

Default: \unit[0.14584]{cm}

\macroname{alterationspace}{}{gsp-default.tex}
Space between an alteration (flat or natural) and the next glyph. 

Default: \unit[0.07747]{cm} plus \unit[0.01276]{cm} minus \unit[0.00455]{cm}

\macroname{beforealterationspace}{}{gsp-default.tex}
Negative space, difference between the normal space between two notes and the space between a note and a flat.  

Default: $\unit[-0.32816]{cm}$ plus \unit[0.01093]{cm} minus \unit[0.01093]{cm}

\macroname{beforelowchoralsignspace}{}{gsp-default.tex}
Space before a low choral sign. 

Default: \unit[0.04556]{cm} plus \unit[0.00638]{cm} minus \unit[0.00638]{cm}

\macroname{clefflatspace}{}{gsp-default.tex}
Space between a clef and a flat (for clefs with flat).  

Default: \unit[0.05469]{cm} plus \unit[0.00638]{cm} minus \unit[0.00638]{cm}

\macroname{interglyphspace}{}{gsp-default.tex}
Space between glyphs in the same element. 

Default: \unit[0.06927]{cm} plus \unit[0.00363]{cm} minus \unit[0.00363]{cm}

\macroname{zerowidthspace}{}{gsp-default.tex}
Null space.  

Default: \unit[0]{cm} plus \unit[0]{cm} minus \unit[0]{cm}

\macroname{interelementspace}{}{gsp-default.tex}
Space between elements.  

Default: \unit[0.06927]{cm} plus \unit[0.00182]{cm} minus \unit[0.00363]{cm}

\macroname{largerspace}{}{gsp-default.tex}
Larger space between elements.  

Default: \unit[0.10938]{cm} plus \unit[0.01822]{cm} minus \unit[0.00911]{cm}

\macroname{glyphspace}{}{gsp-default.tex}
Space between elements which has the size of a note.  

Default: \unit[0.21877]{cm} plus \unit[0.01822]{cm} minus \unit[0.01822]{cm}

\macroname{intersyllablespace}{}{gsp-default.tex}
Minimum space between two notes of different syllables.  

Default: \unit[0.25523]{cm} plus \unit[0.31903]{cm} minus \unit[0]{cm}

\macroname{spacebeforecusto}{}{gsp-default.tex}
Space before custo.  

Default: \unit[0.1823]{cm} plus \unit[0.31903]{cm} minus \unit[0.0638]{cm}

\macroname{spacebeforesigns}{}{gsp-default.tex}
Space before punctum mora and augmentum duplex.  

Default: \unit[0.05469]{cm} plus \unit[0.00455]{cm} minus \unit[0.00455]{cm}

\macroname{spaceaftersigns}{}{gsp-default.tex}
Space after punctum mora and augmentum duplex.  

Default: \unit[0.08203]{cm} plus \unit[0.0082]{cm} minus \unit[0.0082]{cm}

\macroname{spaceafterlineclef}{}{gsp-default.tex}
Space after a clef at the beginning of a line.  

Default: \unit[0.27345]{cm} plus \unit[0.14584]{cm} minus \unit[0.01367]{cm}

\macroname{interwordspacenotes}{}{gsp-default.tex}
Space after at the end of a word when the last written symbol is a note and the first is a note.  

Default: \unit[0.29169]{cm} plus \unit[0.08751]{cm} minus \unit[0.05469]{cm}

\macroname{interwordspacenotestext}{}{gsp-default.tex}
Space after at the end of a word when the last written symbol is a note and the first is text.  

Default: \unit[0.27345]{cm} plus \unit[0.27345]{cm} minus \unit[0.07292]{cm}

\macroname{interwordspacetextnotes}{}{gsp-default.tex}
Space after at the end of a word when the last written symbol is text and the first is a note.  

Default: \unit[0.27345]{cm} plus \unit[0.27345]{cm} minus \unit[0.07292]{cm}

\macroname{interwordspacetext}{}{gsp-default.tex}
Space after at the end of a word when the last written symbol is text and the first is text.  

Default: \unit[0.22787]{cm} plus \unit[0.41019]{cm} minus \unit[0.07292]{cm}

\macroname{bitrivirspace}{}{gsp-default.tex}
Space between notes of a bivirga or trivirga.  

Default: \unit[0.06927]{cm} plus \unit[0.00182]{cm} minus \unit[0.00546]{cm}

\macroname{bitristrospace}{}{gsp-default.tex}
Space between notes of a bistropha or tristrophae.  

Default: \unit[0.06927]{cm} plus \unit[0.00182]{cm} minus \unit[0.00546]{cm}

\macroname{punctuminclinatumshift}{}{gsp-default.tex}
Space between two punctum inclinatum.  

Default: \unit[-0.03918]{cm} plus \unit[0.0009]{cm} minus \unit[0.0009]{cm}

\macroname{beforepunctainclinatashift}{}{gsp-default.tex}
Space before puncta inclinata.  

Default: \unit[0.05286]{cm} plus \unit[0.00728]{cm} minus \unit[0.00455]{cm}

\macroname{punctuminclinatumanddebilisshift}{}{gsp-default.tex}
Space between a punctum inclinatum and a punctum inclinatum deminutus.  

Default: \unit[-0.02278]{cm} plus \unit[0.0009]{cm} minus \unit[0.0009]{cm}

\macroname{punctuminclinatumdebilisshift}{}{gsp-default.tex}
Space between two punctum inclinatum deminutus.  

Default: \unit[-0.00728]{cm} plus \unit[0.0009]{cm} minus \unit[0.0009]{cm}

\macroname{punctuminclinatumbigshift}{}{gsp-default.tex}
Space between puncta inclinata, larger ambitus (range=3rd).  

Default: \unit[0.07565]{cm} plus \unit[0.0009]{cm} minus \unit[0.0009]{cm}

\macroname{punctuminclinatummaxshift}{}{gsp-default.tex}
Space between puncta inclinata, larger ambitus (range=4th -or more?-).  

Default: \unit[0.17865]{cm} plus \unit[0.0009]{cm} minus \unit[0.0009]{cm}

\macroname{spacearoundsmallbar}{}{gsp-default.tex}
Space around virgula and divisio minima.  

Default: \unit[0.1823]{cm} plus \unit[0.22787]{cm} minus \unit[0.05469]{cm}

\macroname{spacearoundminor}{}{gsp-default.tex}
Space around divisio minor.  

Default: \unit[0.1823]{cm} plus \unit[0.22787]{cm} minus \unit[0.05469]{cm}

\macroname{spacearoundmaior}{}{gsp-default.tex}
Space around divisio maior.  

Default: \unit[0.1823]{cm} plus \unit[0.22787]{cm} minus \unit[0.05469]{cm}

\macroname{spacearoundfinalis}{}{gsp-default.tex}
Space around divisio finalis.  

Default: \unit[0.1823]{cm} plus \unit[0.1823]{cm} minus \unit[0.05469]{cm}

\macroname{spacebeforefinalfinalis}{}{gsp-default.tex}
A special space for finalis, for when it is the last glyph.  

Default: \unit[0.29169]{cm} plus \unit[0.07292]{cm} minus \unit[0.27345]{cm}

\macroname{spacearoundclefbars}{}{gsp-default.tex}
Additional space that will appear around bars that are preceded by a custo and followed by a key.  

Default: \unit[0.03645]{cm} plus \unit[0.00455]{cm} minus \unit[0.0009]{cm}

\macroname{textbartextspace}{}{gsp-default.tex}
Space between the text and the text of the bar.  

Default: \unit[0.24611]{cm} plus \unit[0.13672]{cm} minus \unit[0.04921]{cm}

\macroname{notebarspace}{}{gsp-default.tex}
Minimal space between a note and a bar.  

Default: \unit[0.31903]{cm} plus \unit[0.27345]{cm} minus \unit[0.02824]{cm}

\macroname{maximumspacewithoutdash}{}{gsp-default.tex}
Maximal space between two syllables for which we consider a dash is not needed.  

Default: \unit[0.02005]{cm}

\macroname{afterclefnospace}{}{gsp-default.tex}
An extensible space for the beginning of lines.  

Default: \unit[0]{cm} plus \unit[0.27345]{cm} minus \unit[0]{cm}

\macroname{additionalcustoslineswidth}{}{gsp-default.tex}
Width of the additional lines, used only for the custos.  The width is the one for the custos at end of lines, the line for custos in the middle of a score is the same multiplied by 2.  

Default: \unit[0.09114]{cm}

\macroname{afterinitialshift}{}{gsp-default.tex}
Space between the initial and the beginning of the score.  

Default: \unit[0.2457]{cm} plus \unit[0]{cm} minus \unit[0]{cm}

\macroname{beforeinitialshift}{}{gsp-default.tex}
Space between the initial and the beginning of the score.  

Default: \unit[0.2457]{cm} plus \unit[0]{cm} minus \unit[0]{cm}

\macroname{minimalspaceatlinebeginning}{}{gsp-default.tex}
Minimal space at the beginning of a line.

Default: \unit[1.7]{cm}

\macroname{manualinitialwidth}{}{gsp-default.tex}
Space to force the initial width to.  Ignored when 0.  

Default: \unit[0]{cm}

\macroname{aboveinitialseparation}{}{gsp-default.tex}
This space is the one between the bottom of the first annotation line and the top of the second annotation line (above the initial).  

Default: \unit[0.85]{cm}

\macroname{noclefspace}{}{gsp-default.tex}
Space at the beginning of the lines if there is no clef.  

Default: \unit[0.1]{cm}

\macroname{abovesignsspace}{}{gsp-default.tex}
Space above the chironomic signs.  

Default: \unit[0.8]{cm}

\macroname{belowsignsspace}{}{gsp-default.tex}
Space below the chironomic signs.  

Default: \unit[0]{cm}

\macroname{choralsigndownshift}{}{gsp-default.tex}
The distance to shift choral signs down.  The following choral signs are shifted down:

\begin{itemize}
  \item Low choral signs that are not lower than the note
  \item High choral signs which are in a space
  \item Low choral signs that are lower than the note which are in a space
\end{itemize}

Default: \unit[0.00911]{cm}

\macroname{choralsignupshift}{}{gsp-default.tex}
The distance to shift choral signs up.  The following choral signs are shifted up:

\begin{itemize}
  \item High choral signs which are on a line
  \item Low choral signs that are lower than the note which are on a line
\end{itemize}

Default: \unit[0.04556]{cm}

\macroname{translationheight}{}{gsp-default.tex}
The space for the translation.  

Default: \unit[0.5]{cm}

\macroname{spaceabovelines}{}{gsp-default.tex}
The space above the lines.  

Default: \unit[0.45576]{cm} plus \unit[0.36461]{cm} minus \unit[0.09114]{cm}

\macroname{spacelinestext}{}{gsp-default.tex}
The space between the lines and the bottom of the text.  

Default: \unit[0.60617]{cm} plus \unit[0]{cm} minus \unit[0]{cm}

\macroname{spacebeneathtext}{}{gsp-default.tex}
The space beneath the text.  

Default: \unit[0]{cm} plus \unit[0]{cm} minus \unit[0]{cm}

\macroname{abovelinestextraise}{}{gsp-default.tex}
Height of the text above the note line.  

Default: \unit[1.7]{cm}

\macroname{abovelinestextheight}{}{gsp-default.tex}
Height that is added at the top of the lines if there is text above the lines (it must be bigger than the text for it to be taken into consideration).  

Default: \unit[0.3]{cm}

\macroname{braceshift}{}{gsp-default.tex}
An additional shift you can give to the brace above the bars.  

Default: \unit[0]{cm}

\macroname{curlybraceaccentusshift}{}{gsp-default.tex}
A shift you can give to the accentus above the curly brace.  

Default: $\unit[-0.05]{cm}$

%%% Local Variables:
%%% mode: latex
%%% TeX-master: "GregorioRef"
%%% End:
